\section{IVM for the Relational Algebra}\label{sec:relational}

Results in \secref{sec:streams} and~\secref{sec:incremental}
apply to streams of arbitrary group values.  In this
section we apply these results to
IVM for relational databases.  As explained in the introduction, our goal is to
efficiently compute the incremental version of any relational query $Q$
that defines a database view.

However, we face a technical problem: the $\I$ and $\D$ operators were
defined on abelian groups, and relational databases in general are
not abelian groups, since they operate on sets.  Fortunately,
there is a well-known tool in the database literature
which converts set operations into group operations by using \zrs
(also called z-relations~\cite{green-tcs11}) to represent sets.

We start by defining the \zrs group, and then we review how
relational queries are converted into \dbsp circuits  over \zrs.
This translation is efficiently incrementalizable because
many basic relational queries can be expressed using LTI \zr operators~\refsec{sec:relational-operators}.

\subsection{\zrs as an abelian group}

\zrs generalize database tables: think of a \zr as a table where each
row has an associated weight, possibly negative.

Given a set $A$, we define \defined{\zrs} over $A$ as functions with
\emph{finite support} from $A$ to $\Z$.  These are functions $f: A
\rightarrow \Z$ where $f(x) \not= 0$ for at most a finite number of
values $x \in A$.  We also write $\Z[A]$ for the type of \zrs with
elements from $A$.  Values in $\Z[A]$ can be thought of as key-value
maps with keys in $A$ and values in $\Z$, justifying the array
indexing notation.  If $m \in \Z[A]$ we write $m[a]$ instead of
$m(a)$, again using an indexing notation.

A particular \zr $m \in \Z[A]$ can be denoted by enumerating its
elements that have non-zero weights and their corresponding weights:
$m = \{ x_1 \mapsto w_1, \dots, x_n \mapsto w_n \}$.
We call $w_i \in \Z$ the \defined{weight}
of $x_i \in A$.  Weights can be negative.
We write that $x \in m$ iff $m[x] \not= 0$.
We also write $w \cdot x$ for $\{ x \mapsto w \}$.

\ifzsetexamples
Consider a concrete \zr $R \in \Z[\texttt{string}]$,
defined by $R = \{ \texttt{joe} \mapsto 1, \texttt{anne} \mapsto -1 \}$.
$R$ has two elements in its domain,
\texttt{joe} with weight 1 (so $R[\texttt{joe}] = 1$),
and \texttt{anne} with weight $-1$.
We say \texttt{joe} $\in R$ and \texttt{anne} $\in R$.
\fi

Since $\Z$ is an abelian ring, $\Z[A]$ is also an abelian ring (and thus a group).  This group
$(\Z[A], +_{\Z[A]}, 0_{\Z[A]}, -_{\Z{A}})$ has addition and subtraction defined pointwise:
$(f +_{\Z[A]} g)(x) = f(x) + g(x) . \forall x \in A.$
The $0$ element of $\Z[A]$ is the function $0_{\Z[A]}$ defined by $0_{\Z[A]}(x) = 0 .
\forall x \in A$.  For example, $R + R =  \{ \texttt{joe} \mapsto 2, \texttt{anne} \mapsto -2 \}$.
Since \zrs form a group, all results from \secref{sec:streams} apply to streams over \zrs.

\zrs generalize sets and bags.  A set with elements from $A$
can be represented as a \zr by associating a weight of 1 with each element.
Bags are \zrs where all weights are positive.  Crucially, \zrs
can also represent arbitrary \emph{changes} to sets and bags.
Negative weights in a change represent elements that are being ``removed''.

\begin{definition}
We say that a \zr represents a \defined{set} if the weight of every
element is one.  We define a function to check this property
$\isset : \Z[A] \rightarrow \B$\index{isset}
given by:
$$\isset(m) \defn \left\{
\begin{array}{ll}
  \mbox{true} & \mbox{ if } m[x] = 1, \forall x \in m \\
  \mbox{false} & \mbox{ otherwise}
\end{array}
\right.
$$
\end{definition}

\ifzsetexamples
For our example $\isset(R) = \mbox{false}$, since $R[\texttt{anne}] = -1$.
\fi

\begin{definition}
We say that a \zr is \defined{positive} (or a \defined{bag}) if the weight of every element is
positive.  We define a function to check this property
$\ispositive : \Z[A] \rightarrow \B$\index{ispositive}.
given by
$$\ispositive(m) \defn \left\{
\begin{array}{ll}
  \mbox{true} & \mbox{ if } m[x] \geq 0, \forall x \in A \\
  \mbox{false} & \mbox{ otherwise}
\end{array}
\right.$$
\end{definition}
We have $\forall m \in \Z[A] . \isset(m) \Rightarrow \ispositive(m)$.
\ifzsetexamples
$\ispositive(R) = \mbox{false}$, since $R[\texttt{anne}] = -1$.
\fi

We write $m \geq 0$ when $m$ is positive.  For positive $m, n \in
\Z[A]$ we write $m \geq n$ for iff $m - n \geq 0$.  $\geq$ is a
partial order.

We call a function $f : \Z[A] \rightarrow \Z[B]$ \defined{positive} if it maps
positive values to positive values:
$\forall x \in \Z[A], x \geq 0_{\Z[A]} \Rightarrow f(x) \geq 0_{\Z[B]}$.
We use the same notation for functions: $\ispositive(f)$.

\begin{definition}[distinct]
The function $\distinct: \Z[A] \rightarrow \Z[A]$\index{distinct}
``converts'' a \zr into a set:
$$\distinct(m)[x] \defn \left\{
\begin{array}{ll}
  1 & \mbox{ if } m[x] > 0 \\
  0 & \mbox{ otherwise}
\end{array}
\right.
$$
\end{definition}

Notice that $\distinct$ ``removes'' duplicates from multisets, and it also eliminates
elements with negative weights.
\ifzsetexamples
$\distinct(R) = \{ \texttt{joe} \mapsto 1 \}$.
\fi
While very simple, this definition of $\distinct$ has been carefully
chosen to enable us to implement the relational (set) operators
using \zrs operators.
%Circuits derived from relational queries only compute on positive \zrs.

%\begin{definition}(mononotonicity)
%A stream $s \in \stream{\Z[A]}$ is \defined{positive} if every value of the stream is positive:
%$s[t] \geq 0 . \forall t \in \N$.
%A stream $s \in \stream{\Z[A]}$ is \defined{monotone} if $s[t] \geq s[t-1], \forall t \in \N$.
%\end{definition}
%
%If $s \in \stream{\Z[A]}$ is positive, then $\I(s)$ is monotone.
%If $s \in \stream{\Z[A]}$ is monotone, $\D(s)$ is positive.
%
\paragraph{Generalizing circuit diagrams}

From now on we will use circuits to compute both on scalars (\zrs in our case) and streams of \zrs.
We use the same graphical representation for functions on streams or scalars:
boxes with input and output arrows.  For scalar functions the ``values''
of the arrows are scalars instead of streams; otherwise
the interpretation of boxes as function application is unchanged.  We will
thus use circuits to depict relational query plans.

\subsection{Implementing relational operators}\label{sec:relational-operators}

The fact that relational algebra can be implemented by computations
on \zrs has been shown before, e.g.~\cite{green-pods07}.  The translation
of the relational operators to \dbsp is shown in Table~\ref{tab:relational}.
The first row of the table shows that a composite query is translated
recursively.  This gives us a recipe for
translating an arbitrary relational query plan into a \dbsp circuit.

\newlength{\commentsize}
\setlength{\commentsize}{5cm}
\begin{table*}[h]
\small
\begin{tabular}{|m{1.2cm}m{4.2cm}m{5cm}m{\commentsize}|} \hline
Operation & SQL example & \dbsp circuit & Details \\ \hline
Composition &
 \begin{lstlisting}[language=SQL]
SELECT DISTINCT ... FROM
(SELECT ... FROM ...)
\end{lstlisting}
 &
 \begin{tikzpicture}[auto,>=latex]
  \node[] (I) {\code{I}};
  \node[block, right of=I] (CI) {$C_I$};
  \draw[->] (I) -- (CI);
  \node[block, right of=CI] (CO) {$C_O$};
  \node[right of=CO] (O) {\code{O}};
  \draw[->] (CI) -- (CO);
  \draw[->] (CO) -- (O);
\end{tikzpicture}
 &
 \parbox[b][][t]{\commentsize}{
  $C_I$ circuit for inner query, \\
  $C_O$ circuit for outer query.}
\\ \hline
Union &
\begin{lstlisting}[language=SQL]
(SELECT * FROM I1)
UNION
(SELECT * FROM I2)
\end{lstlisting}
&
\begin{tikzpicture}[auto,>=latex]
  \node[] (input1) {\code{I1}};
  \node[below of=input1, node distance=.4cm] (midway) {};
  \node[below of=midway, node distance=.4cm] (input2) {\code{I2}};
  \node[block, shape=circle, right of=midway, inner sep=0in] (plus) {$+$};
  \node[block, right of=plus] (distinct) {$\distinct$};
  \node[right of=distinct] (output) {\code{O}};
  \draw[->] (input1) -| (plus);
  \draw[->] (input2) -| (plus);
  \draw[->] (plus) -- (distinct);
  \draw[->] (distinct) -- (output);
\end{tikzpicture}
& $\distinct$ eliminates duplicates.  An implementation of
\texttt{UNION ALL} does not need the $\distinct$.
\\ \hline
Projection &
\begin{lstlisting}[language=SQL]
SELECT DISTINCT I.c
FROM I
\end{lstlisting}
&
\begin{tikzpicture}[auto,>=latex]
  \node[] (input) {\code{I}};
  \node[block, right of=input] (pi) {$\pi$};
  \node[block, right of=pi] (distinct) {$\distinct$};
  \node[right of=distinct] (output) {\code{O}};
  \draw[->] (input) -- (pi);
  \draw[->] (pi) -- (distinct);
  \draw[->] (distinct) -- (output);
\end{tikzpicture}
&
\parbox[b][][t]{\commentsize}{
$\pi(i)[y] \defn
\sum_{x \in i, x|_c = y} i[x]$ \\
$x|_c$ is projection on column $c$ of the tuple $x$ \\
$\pi$ is linear; $\ispositive(\pi)$ %, \zpp{\pi}$.
}
\\ \hline
Filtering &
\begin{lstlisting}[language=SQL]
SELECT * FROM I
WHERE p(I.c)
\end{lstlisting}
&
\begin{tikzpicture}[auto,>=latex]
  \node[] (input) {\code{I}};
  \node[block, right of=input] (map) {$\sigma_P$};
  \node[right of=map] (output) {\code{O}};
  \draw[->] (input) -- (map);
  \draw[->] (map) -- (output);
\end{tikzpicture}
&
\parbox[b][][t]{\commentsize}{
$\sigma_P(m)[x] \defn \left\{
\begin{array}{ll}
  m[x] & \mbox{ if } P(x) \\
  0 & \mbox{ otherwise } \\
\end{array}
\right.$ \\
$P: A \rightarrow \B$ is a predicate. \\
$\sigma_P$ is linear; $\ispositive(\sigma_P)$ % \zpp{\sigma_P}$.
}
% \\ \hline
%Selection &
%\begin{lstlisting}[language=SQL]
%SELECT DISTINCT f(I.c, ...)
%FROM I
%\end{lstlisting}
%&
%\begin{tikzpicture}[auto,>=latex]
%  \node[] (input) {\code{I}};
%  \node[block, right of=input, node distance=1.5cm] (map) {$\mbox{map}(f)$};
%  \node[block, right of=map, node distance=1.5cm] (distinct) {$\distinct$};
%  \node[right of=distinct, node distance=1.5cm] (output) {\code{O}};
%  \draw[->] (input) -- (map);
%  \draw[->] (map) -- (distinct);
%  \draw[->] (distinct) -- (output);
%\end{tikzpicture}
%&
%\parbox[b][][t]{\commentsize}{
%For a function $f$ \\
%$\map(f)$ is linear, \\
%$\ispositive(\map(f)), \zpp{\map(f)}$
%}.
\\ \hline
\parbox[b][][t]{1cm}{
Cartesian \\
product} &
\begin{lstlisting}[language=SQL]
SELECT I1.*, I2.*
FROM I1, I2
\end{lstlisting}
&
\begin{tikzpicture}[auto,>=latex]
  \node[] (i1) {\code{I1}};
  \node[below of=i1, node distance=.4cm] (midway) {};
  \node[below of=midway, node distance=.4cm] (i2) {\code{I2}};
  \node[block, right of=midway] (prod) {$\times$};
  \node[right of=prod] (output) {\code{O}};
  \draw[->] (i1) -| (prod);
  \draw[->] (i2) -| (prod);
  \draw[->] (prod) -- (output);
\end{tikzpicture}
&
\parbox[b][][t]{\commentsize}{
$(a \times b)((x,y)) \defn a[x] \times b[y]$. \\
$\times$ is bilinear, $\ispositive(\times)$ % , \zpp{\times}$.
}
\\ \hline
Equi-join &
\begin{lstlisting}[language=SQL]
SELECT I1.*, I2.*
FROM I1 JOIN I2
ON I1.c1 = I2.c2
\end{lstlisting}
&
\begin{tikzpicture}[auto,>=latex]
  \node[] (i1) {\code{I1}};
  \node[below of=i1, node distance=.4cm] (midway) {};
  \node[below of=midway, node distance=.4cm] (i2) {\code{I2}};
  \node[block, right of=midway] (prod) {$\bowtie_{c1 = c2}$};
  \node[right of=prod] (output) {\code{O}};
  \draw[->] (i1) -| (prod);
  \draw[->] (i2) -| (prod);
  \draw[->] (prod) -- (output);
\end{tikzpicture}
&
\parbox[b][][t]{\commentsize}{
$(a \bowtie b)((x,y)) \defn a[x] \times b[y] \\
\mbox{ if } x|_{c1} = y|_{c2}$. \\
$\bowtie$ is bilinear, $\ispositive(\bowtie)$ %, \zpp{\bowtie}$.
}
\\ \hline
Intersection &
\begin{lstlisting}[language=SQL]
(SELECT * FROM I1)
INTERSECT
(SELECT * FROM I2)
\end{lstlisting}
&
\begin{tikzpicture}[auto,>=latex]
  \node[] (i1) {\code{I1}};
  \node[below of=i1, node distance=.4cm] (midway) {};
  \node[below of=midway, node distance=.4cm] (i2) {\code{I2}};
  \node[block, right of=midway] (prod) {$\bowtie$};
  \node[right of=prod] (output) {\code{O}};
  \draw[->] (i1) -| (prod);
  \draw[->] (i2) -| (prod);
  \draw[->] (prod) -- (output);
\end{tikzpicture}
&
Special case of equi-join when both relations have the same schema.
 \\ \hline
Difference &
\begin{lstlisting}[language=SQL]
SELECT * FROM I1
EXCEPT
SELECT * FROM I2
\end{lstlisting}
&
\begin{tikzpicture}[auto,>=latex, node distance=.7cm]
  \node[] (i1) {\code{I1}};
  \node[below of=i1, node distance=.4cm] (midway) {};
  \node[below of=midway, node distance=.4cm] (i2) {\code{I2}};
  \node[block, shape=circle, inner sep=0in, right of=i2] (m) {$-$};
  \node[block, right of=midway, shape=circle, inner sep=0in, node distance=1.3cm] (plus) {$+$};
  \node[block, right of=plus, node distance=1cm] (distinct) {$\distinct$};
  \node[right of=distinct, node distance=1cm] (output) {\code{O}};
  \draw[->] (i1) -| (plus);
  \draw[->] (i2) -- (m);
  \draw[->] (m) -| (plus);
  \draw[->] (plus) -- (distinct);
  \draw[->] (distinct) -- (output);
\end{tikzpicture}
&
$\distinct$ removes elements with negative weights from the result.
\\ \hline
\end{tabular}
\caption{Implementation of SQL relational set operators in \dbsp.
Each query assumes that inputs \code{I}, \code{I1}, \code{I2}, are sets and it
produces output sets.\label{tab:relational}}
\end{table*}


The translation is fairly straightforward, but many operators require
the application of a $\distinct$ \textbf{to produce sets}.
For example, $a \cup b = \distinct(a + b)$, $a \setminus b =
\distinct(a - b)$, $(a \times b)((x,y)) = a[x] \times b[y]$.
%\paragraph{Correctness of the \dbsp implementations}\label{sec:correctness}
%
%A relational query $Q$ that transforms
%a set $V$ into a set $U$ is implemented by a \dbsp computation $Q'$ on
%\zrs.  The correctness of the implementation requires the following
%diagram to commute:
%
%\begin{center}
%\begin{tikzpicture}
%  \node[] (V) {$V$};
%  \node[below of=V] (VZ) {$VZ$};
%  \node[right of=V, node distance=2cm] (U) {$U$};
%  \node[below of=U] (UZ) {$UZ$};
%  \draw[->] (V) -- node (f) [below] {$Q$} (U);
%  \draw[->] (V) --  node (s) [left] {tozset}(VZ);
%  \draw[->] (VZ) -- node (f) [above] {$Q'$} (UZ);
%  \draw[->] (UZ) -- node (d) [right] {toset} (U);
%\end{tikzpicture}
%\end{center}
%
%(The correctness of
%this implementation is predicated on $Q'$'s inputs being
%sets, an invariant which needs to be maintained by the environment.)
%The ``$\mbox{toset}$'' and ``$\mbox{tozset}$'' functions convert sets to \zrs and
%vice-versa, in the expected way:
%
%$\mbox{toset}: \Z[A] \to 2^A$ is defined as $\mbox{toset}(m) \defn \cup_{x \in \distinct(m)} \{ x \}$.
%
%$\mbox{tozset}: 2^A \to \Z[A]$ is defined as $\mbox{tozset}(s) \defn \sum_{x \in s} 1 \cdot x$.
%
%All standard algebraic properties
%of the relational operators can be used to optimize circuits
%(they can even be applied to queries before building the circuits).
%
Notice that the use of the $\distinct$ operator allows \dbsp to model
the \emph{full relational algebra}, including set difference (and not
just the positive fragment).

Prior work (e.g., Proposition 6.13 in~\cite{green-tcs11}) has shown
how some invocations of $\distinct$ can be eliminated from query plans
without changing the query semantics; we will see that incremental
versions of $\distinct$ operators incur significant space costs.

\begin{proposition}\label{prop-distinct-delay}
Let $Q$ be one of the following \zrs operators: filtering $\sigma$,
join $\bowtie$, or Cartesian product $\times$.
Then we have $\forall i \in \Z[I], \ispositive(i) \Rightarrow Q(\distinct(i)) = \distinct(Q(i))$.
\end{proposition}

\begin{comment}
\noindent
\begin{tabular}{m{3.5cm}m{.5cm}m{3.5cm}}
\begin{tikzpicture}[auto,>=latex]
  \node[] (input) {$i$};
  \node[block, right of=input, node distance=1.1cm] (distinct) {$\distinct$};
  \node[block, right of=distinct, node distance=1.2cm] (q) {$Q$};
  \node[right of=q] (output)  {$o$};
  \draw[->] (input) -- (distinct);
  \draw[->] (distinct) -- (q);
  \draw[->] (q) -- (output);
\end{tikzpicture}
&
$\cong$
&
\begin{tikzpicture}[auto,>=latex]
  \node[] (input) {$i$};
  \node[block, right of=input] (q) {$Q$};
  \node[block, right of=q, node distance=1.2cm] (distinct1) {$\distinct$};
  \node[right of=distinct1, node distance=1.2cm] (output)  {$o$};
  \draw[->] (input) -- (q);
  \draw[->] (q) -- (distinct1);
  \draw[->] (distinct1) -- (output);
\end{tikzpicture}
\end{tabular}

This rule allows us to delay the application of $\distinct$.
\end{comment}

\begin{proposition}\label{prop-distinct-once}
Let $Q$ be one of the following \zrs operators: filtering $\sigma$,
projection $\pi$, map($f$)\footnote{Technically, map (applying a user-defined
function to each row) is not relational.},
addition $+$, join $\bowtie$, or
Cartesian product $\times$.
Then we have $\ispositive(i) \Rightarrow \distinct(Q(\distinct(i))) = \distinct(Q(i))$.
\end{proposition}

\begin{comment}
\noindent
\begin{tabular}{m{6.5cm}m{.5cm}}
\begin{tikzpicture}[auto,>=latex]
  \node[] (input) {$i$};
  \node[block, right of=input, node distance=1.5cm] (distinct) {$\distinct$};
  \node[block, right of=distinct, node distance=1.5cm] (q) {$Q$};
  \node[block, right of=q, node distance=1.5cm] (distinct1) {$\distinct$};
  \node[right of=distinct1, node distance=1.5cm] (output)  {$o$};
  \draw[->] (input) -- (distinct);
  \draw[->] (distinct) -- (q);
  \draw[->] (q) -- (distinct1);
  \draw[->] (distinct1) -- (output);
\end{tikzpicture}
&
$\cong$ \\
\begin{tikzpicture}[auto,>=latex]
  \node[] (input) {$i$};
  \node[block, right of=input] (q) {$Q$};
  \node[block, right of=q, node distance=1.5cm] (distinct1) {$\distinct$};
  \node[right of=distinct1, node distance=1.5cm] (output)  {$o$};
  \draw[->] (input) -- (q);
  \draw[->] (q) -- (distinct1);
  \draw[->] (distinct1) -- (output);
\end{tikzpicture}
\end{tabular}
\end{comment}

These properties allow us to ``consolidate'' distinct operators by performing
one $\distinct$ at the end of a chain of computations.

\subsection{Incremental view maintenance}

Let us consider a relational query $Q$ defining a view $V$.  To create
a circuit that maintains incrementally $V$ we apply the following
mechanical steps:

\begin{algorithm}[incremental view maintenance]\label{algorithm-inc}\quad
\begin{enumerate}[nosep, leftmargin=\parindent]
    \item Translate $Q$ into a circuit using the rules in Table~\ref{tab:relational}.
    \item Apply $\distinct$ elimination rules (\ref{prop-distinct-delay}, \ref{prop-distinct-once}) until convergence\footnote{The
    order in which the rules are applied does not matter, since the algorithm is
    confluent: it always produces the same final result.}.
    \item Lift the whole circuit, by applying Proposition~\ref{prop:distributivity},
    converting it to a circuit operating on streams.
    \item Incrementalize the whole circuit ``surrounding'' it with $\I$ and $\D$.
    \item Apply the chain rule
    from Proposition~\ref{prop-inc-properties} recursively on the query structure
    to obtain an incremental implementation.
\end{enumerate}
\end{algorithm}

This algorithm is deterministic and its running time
is proportional to the number of operators in the query.
Step (2) generates an equivalent circuit, with possibly fewer
$\distinct$ operators.  Step (3) yields a circuit that consumes a
\emph{stream} of complete database snapshots and outputs a stream of
complete view snapshots. Step (4) yields a circuit that consumes a
stream of \emph{database changes} and outputs a stream of \emph{view
changes}; however, the internal operation of the circuit is
non-incremental, as it rebuilds the complete database using
integration operators.  Step (5) incrementalizes the circuit by
replacing each primitive operator with its incremental version.

Most of the operators that appear in the circuits in
Table~\ref{tab:relational} are linear, and thus have very efficient
incremental versions (we discuss complexity in
\refsec{sec:complexity}).  A notable exception is $\distinct$.  The
next proposition shows that the incremental version of $\distinct$ is
also efficient, and it can be computed by doing work proportional to
the size of the input change:

\begin{proposition}\label{prop-inc_distinct}
The following circuit implements $\inc{(\lift{\distinct})}$:
\begin{tabular}{m{3.5cm}m{.0cm}m{5cm}}
\begin{tikzpicture}[auto,node distance=1.5cm,>=latex]
    \node[] (input) {$\Delta d$};
    \node[block, right of=input] (d) {$\inc{(\lift{\distinct})}$};
    \node[right of=d] (output) {$\Delta o$};
    \draw[->] (input) -- (d);
    \draw[->] (d) -- (output);
\end{tikzpicture} &
$\cong$ &
\begin{tikzpicture}[>=latex]
    \node[] (input) {$\Delta d$};
    \node[block, right of=input] (I) {$\I$};
    \node[block, right of=I] (z) {$\zm$};
    \node[block, below of=z, node distance=.8cm] (H) {$\lift{H}$};
    \node[right of=H] (output) {$\Delta o$};
    \draw[->] (input) -- node (mid) {} (I);
    \draw[->] (I) -- (z);
    \draw[->] (mid.center) |- (H);
    \draw[->] (z) -- node (i) [right] {$i$} (H);
    \draw[->] (H) -- (output);
\end{tikzpicture}
\end{tabular}

\noindent where $H: \Z[A] \times \Z[A] \to \Z[A]$ is defined as: \\
$$
H(i, d)[x] \defn
\begin{cases}
-1 & \mbox{if } i[x] > 0 \mbox{ and } (i + d)[x] \leq 0 \\
1  & \mbox{if } i[x] \leq 0 \mbox{ and } (i + d)[x] > 0 \\
0  & \mbox{otherwise} \\
\end{cases}
$$
\end{proposition}

Here is the intuition why $\distinct$ is efficiently
incrementalizable: the only elements that can appear in the output of
$\inc{(\lift{\distinct})}$ must have changed in the input.  So the
size of the output change cannot be bigger than the size of the input
change.  In the diagram above, $i$ is the previous version of the
integral of all changes, i.e., the full \zr whose $\distinct$ value is
being computed.  The function $H$ detects whether the weight of
an element in $i$ is changing sign (positive to negative or
vice-versa) when adding a new delta $d$.

%\refsec{sec:relational-example} shows a concrete example of a relational query converted
%into a circuit and then incrementalized using Algorithm~\ref{algorithm-inc}.

\subsection{Complexity of incremental circuits}\label{sec:complexity}

Incremental circuits are efficient.  We evaluate the cost of a circuit
while processing the $t$-th input change.  Even if $Q$ is a pure
function, $\inc{Q}$ is actually a streaming system, with internal
state.  This state is stored entirely in the delay operators $\zm$,
some of which appear in $\I$ and $\D$ operators.  The result produced
by $\inc{Q}$ on the $t$-th input depends in general not only on the
new $t$-th input, but also on all prior inputs it has received.

We argue that each operator in the incremental version of a circuit is
efficient in terms of work and space.  We make the standard IVM
assumption that the input changes \emph{of each operator} are small:
$|\Delta DB[t]| \ll |DB[t]| = |(\I(\Delta DB))[t]|$.

An unoptimized incremental operator $\inc{Q} = \D \circ Q \circ \I$
evaluates query $Q$ on the whole database $DB$, the integral of the input stream:
$DB = \I(\Delta DB)$; hence its time complexity  is the same as that of the non-incremental
evaluation of $Q$.  In addition, each of the $\I$ and $\D$ operators uses $O(|DB[t]|)$ memory.

Step (5) of the incrementalization algorithm applies the optimizations described in \secref{sec:incremental};
these reduce the time complexity of each operator to be a function of $O(|\Delta DB[t]|)$.
For example, Theorem~\ref{linear}, allows evaluating $\inc{S}$, where $S$ is a
linear operator, in time $O(|\Delta DB[t]|)$.  The $\I$
operator can also be evaluated in $O(|\Delta DB[t]|)$ time, because
all values that appear in the output of $\I(\Delta DB)[t]$ must be present in
current input change $\Delta DB[t]$.  Similarly, while the $\distinct$ operator is not
linear, $\inc{(\lift{\distinct})}$ can also be evaluated in $O(|\Delta DB[t]|)$ according to
Proposition~\ref{prop-inc_distinct}.  Bilinear operators, including join, can be
evaluated in time $O(|DB[t]| \times |\Delta DB[t]|)$, which is a factor of $|DB[t] / \Delta DB[t]|$
better than full re-evaluation.

The space complexity of linear operators is 0 (zero), since they store no
data persistently.  The space complexity of operators such as $\inc{(\lift{\distinct})}$,
$\inc{(\lift{\bowtie})}$, $\I$, and $\D$ is $O(|DB[t]|)$.  They need
to store their input or output relations in full.

\subsubsection{IVM query plans and optimality}

Let us look again at what we achieved using
Algorithm~\ref{algorithm-inc}.  A relational algebra query can be
implemented by multiple plans, each with a different data-dependent
cost\footnote{The optimal plan depends not only on the query, but also
  on the data.}.  The input of Algorithm~\ref{algorithm-inc} is a
(relational), non-incremental query plan, produced by a query planner.
The algorithm produces an incremental plan that is ``similar'' to the
input plan.

Standard query planners use cost-based heuristics and data statistics
to optimize plans.  A generic IVM planner does not have this luxury,
since the plan must be generated \emph{before} any data has been fed
to the database.  Nevertheless, all standard query optimization
techniques, perhaps based on historical statistics, can be used to
generate the query plan that is supplied to
Algorithm~\ref{algorithm-inc}.  The question of optimality in the
context of IVM plan is a much more difficult topic than optimization
of ad-hoc queries, since the chosen IVM plan will execute for
\emph{all future database updates}.

Moreover, since incremental computations maintain internal state, it
follows that incremental plans cannot be simply changed in-flight,
like we can change ad-hoc queries based on current data statistics:
deploying a new plan requires in general constructing its internal
state, which is produced by entire history of prior updates.
Fortunately, there is a trivial, but somewhat expensive, recipe for
installing a new incremental plan: feed the entire current state of
the database, as one big change.

\subsection{Relational Query Example}\label{sec:relational-example}

We apply the IVM algorithm~\ref{algorithm-inc} to a concrete
relational SQL query:
\begin{lstlisting}[language=SQL,basicstyle=\small]
CREATE VIEW v AS
SELECT DISTINCT a.x, b.y FROM (
     SELECT t1.x, t1.id FROM t1 WHERE t1.a > 2
) a JOIN (
     SELECT t2.id, t2.y FROM t2 WHERE t2.s > 5
) b ON a.id = b.id
\end{lstlisting}

Step 1: Create a \dbsp circuit to represent this query using the
translation rules from Table~\ref{tab:relational}; notice that
this circuit is essentially a dataflow implementation of the query.
(Notice that the query asks for \code{SELECT DISTINCT}, so there is a
$\distinct$ operator after $\sigma$):

\noindent
\begin{tikzpicture}[node distance=1.2cm,>=latex]
    \node[] (t1) {\code{t1}};
    \node[block, right of=t1, node distance=.9cm] (s1) {$\sigma_{a > 2}$};
    \node[block, right of=s1] (d1) {$\distinct$};
    \node[block, right of=d1] (p1) {$\pi_{x, id}$};
    \node[block, right of=p1] (d11) {$\distinct$};
    \node[below of=t1, node distance=1cm] (t2) {\code{t2}};
    \node[block, right of=t2, node distance=.9cm] (s2) {$\sigma_{s > 5}$};
    \node[block, right of=s2] (d2) {$\distinct$};
    \node[block, right of=d2] (p2) {$\pi_{y, id}$};
    \node[block, right of=p2] (d21) {$\distinct$};
    \node[below of=d11, node distance=.5cm] (mid) {};
    \node[block, right of=mid, node distance=.8cm] (j) {$\bowtie_{id = id}$};
    \node[block, right of=j] (p) {$\pi_{x, y}$};
    \node[block, right of=p] (d) {$\distinct$};
    \node[right of=d, node distance=.9cm] (V) {\code{V}};
    \draw[->] (t1) -- (s1);
    \draw[->] (s1) -- (d1);
    \draw[->] (d1) -- (p1);
    \draw[->] (p1) -- (d11);
    \draw[->] (t2) -- (s2);
    \draw[->] (s2) -- (d2);
    \draw[->] (d2) -- (p2);
    \draw[->] (p2) -- (d21);
    \draw[->] (d11) -| (j);
    \draw[->] (d21) -| (j);
    \draw[->] (j) -- (p);
    \draw[->] (p) -- (d);
    \draw[->] (d) -- (V);
\end{tikzpicture}

Step 2: apply the rules to eliminate $\distinct$ operators.
First from Proposition~\ref{prop-distinct-once}:

\noindent
\begin{tikzpicture}[node distance=1.2cm,>=latex]
    \node[] (t1) {\code{t1}};
    \node[block, right of=t1, node distance=.9cm] (s1) {$\sigma_{a > 2}$};
    \node[block, right of=s1] (p1) {$\pi_{x, id}$};
    \node[block, right of=p1] (d11) {$\distinct$};
    \node[below of=t1, node distance=1cm] (t2) {\code{t2}};
    \node[block, right of=t2, node distance=.9cm] (s2) {$\sigma_{s > 5}$};
    \node[block, right of=s2] (p2) {$\pi_{y, id}$};
    \node[block, right of=p2] (d21) {$\distinct$};
    \node[below of=d11, node distance=.5cm] (mid) {};
    \node[block, right of=mid, node distance=.8cm] (j) {$\bowtie_{id = id}$};
    \node[block, right of=j] (p) {$\pi_{x, y}$};
    \node[block, right of=p] (d) {$\distinct$};
    \node[right of=d, node distance=.9cm] (V) {\code{V}};
    \draw[->] (t1) -- (s1);
    \draw[->] (s1) -- (p1);
    \draw[->] (p1) -- (d11);
    \draw[->] (t2) -- (s2);
    \draw[->] (s2) -- (p2);
    \draw[->] (p2) -- (d21);
    \draw[->] (d11) -| (j);
    \draw[->] (d21) -| (j);
    \draw[->] (j) -- (p);
    \draw[->] (p) -- (d);
    \draw[->] (d) -- (V);
\end{tikzpicture}

\noindent The rule from Proposition~\ref{prop-distinct-delay} gives
(from now on we omit the subscripts to save space):

\noindent
\begin{tikzpicture}[node distance=1.2cm,>=latex]
    \node[] (t1) {\code{t1}};
    \node[block, right of=t1, node distance=.9cm] (s1) {$\sigma$};
    \node[block, right of=s1] (p1) {$\pi$};
    \node[below of=t1, node distance=1cm] (t2) {\code{t2}};
    \node[block, right of=t2, node distance=.9cm] (s2) {$\sigma$};
    \node[block, right of=s2] (p2) {$\pi$};
    \node[below of=p1, node distance=.5cm] (mid) {};
    \node[block, right of=mid, node distance=.8cm] (j) {$\bowtie$};
    \node[block, right of=j] (d0) {$\distinct$};
    \node[block, right of=d0] (p) {$\pi$};
    \node[block, right of=p] (d) {$\distinct$};
    \node[right of=d, node distance=.9cm] (V) {\code{V}};
    \draw[->] (t1) -- (s1);
    \draw[->] (s1) -- (p1);
    \draw[->] (t2) -- (s2);
    \draw[->] (s2) -- (p2);
    \draw[->] (p1) -| (j);
    \draw[->] (p2) -| (j);
    \draw[->] (j) -- (d0);
    \draw[->] (d0) -- (p);
    \draw[->] (p) -- (d);
    \draw[->] (d) -- (V);
\end{tikzpicture}

\noindent And again~\ref{prop-distinct-once}:

\noindent
\begin{tikzpicture}[node distance=1cm,>=latex]
    \node[] (t1) {\code{t1}};
    \node[block, right of=t1, node distance=.9cm] (s1) {$\sigma$};
    \node[block, right of=s1] (p1) {$\pi$};
    \node[below of=t1, node distance=1cm] (t2) {\code{t2}};
    \node[block, right of=t2, node distance=.9cm] (s2) {$\sigma$};
    \node[block, right of=s2] (p2) {$\pi$};
    \node[below of=p1, node distance=.5cm] (mid) {};
    \node[block, right of=mid, node distance=.8cm] (j) {$\bowtie$};
    \node[block, right of=j] (p) {$\pi$};
    \node[block, right of=p] (d) {$\distinct$};
    \node[right of=d, node distance=1cm] (V) {\code{V}};
    \draw[->] (t1) -- (s1);
    \draw[->] (s1) -- (p1);
    \draw[->] (t2) -- (s2);
    \draw[->] (s2) -- (p2);
    \draw[->] (p1) -| (j);
    \draw[->] (p2) -| (j);
    \draw[->] (j) -- (p);
    \draw[->] (p) -- (d);
    \draw[->] (d) -- (V);
\end{tikzpicture}

At this point no more $\distinct$ elimination rules can be applied.

Step 3: we lift the circuit using distributivity of composition over
lifting (Proposition~\ref{prop:distributivity}); we obtain a circuit
that computes over streams, i.e., for each new input pair of relations
\code{t1} and \code{t2} it will produce an output view \code{V}:

\noindent
\begin{tikzpicture}[node distance=1cm,>=latex]
    \node[] (t1) {\code{t1}};
    \node[block, right of=t1, node distance=.9cm] (s1) {$\lift{\sigma}$};
    \node[block, right of=s1] (p1) {$\lift{\pi}$};
    \node[below of=t1, node distance=1cm] (t2) {\code{t2}};
    \node[block, right of=t2, node distance=.9cm] (s2) {$\lift{\sigma}$};
    \node[block, right of=s2] (p2) {$\lift{\pi}$};
    \node[below of=p1, node distance=.5cm] (mid) {};
    \node[block, right of=mid, node distance=.8cm] (j) {$\lift{\bowtie}$};
    \node[block, right of=j] (p) {$\lift{\pi}$};
    \node[block, right of=p, node distance=1.2cm] (d) {$\lift{\distinct}$};
    \node[right of=d] (V) {\code{V}};
    \draw[->] (t1) -- (s1);
    \draw[->] (s1) -- (p1);
    \draw[->] (t2) -- (s2);
    \draw[->] (s2) -- (p2);
    \draw[->] (p1) -| (j);
    \draw[->] (p2) -| (j);
    \draw[->] (j) -- (p);
    \draw[->] (p) -- (d);
    \draw[->] (d) -- (V);
\end{tikzpicture}

Step 4: incrementalize circuit, obtaining a circuit that computes over changes;
this circuit receives changes to relations \code{t1} and \code{t2} and for each
such change it produces the corresponding change in the output view \code{V}:

\noindent
\begin{tikzpicture}[node distance=1cm,>=latex]
    \node[] (t1) {$\Delta$\code{t1}};
    \node[block, right of=t1, node distance=.8cm] (I1) {$\I$};
    \node[block, right of=I1, node distance=.9cm] (s1) {$\lift{\sigma}$};
    \node[block, right of=s1] (p1) {$\lift{\pi}$};
    \node[below of=t1, node distance=1cm] (t2) {$\Delta$\code{t2}};
    \node[block, right of=t2, node distance=.8cm] (I2) {$\I$};
    \node[block, right of=I2, node distance=.9cm] (s2) {$\lift{\sigma}$};
    \node[block, right of=s2] (p2) {$\lift{\pi}$};
    \node[below of=p1, node distance=.5cm] (mid) {};
    \node[block, right of=mid, node distance=.7cm] (j) {$\lift{\bowtie}$};
    \node[block, right of=j] (p) {$\lift{\pi}$};
    \node[block, right of=p, node distance=1.2cm] (d) {$\lift{\distinct}$};
    \node[block, right of=d, node distance=1.1cm] (D) {$\D$};
    \node[right of=D, node distance=.8cm] (V) {$\Delta$\code{V}};
    \draw[->] (t1) -- (I1);
    \draw[->] (I1) -- (s1);
    \draw[->] (s1) -- (p1);
    \draw[->] (t2) -- (I2);
    \draw[->] (I2) -- (s2);
    \draw[->] (s2) -- (p2);
    \draw[->] (p1) -| (j);
    \draw[->] (p2) -| (j);
    \draw[->] (j) -- (p);
    \draw[->] (p) -- (d);
    \draw[->] (d) -- (D);
    \draw[->] (D) -- (V);
\end{tikzpicture}

Step 5: apply the chain rule to rewrite the circuit as a composition of incremental operators;

\noindent
\begin{tikzpicture}[node distance=1.6cm,>=latex]
    \node[] (t1) {$\Delta$\code{t1}};
    \node[block, right of=t1, node distance=1.2cm] (s1) {$\inc{(\lift{\sigma})}$};
    \node[block, right of=s1] (p1) {$\inc{(\lift{\pi})}$};
    \node[below of=t1, node distance=1.2cm] (t2) {$\Delta$\code{t2}};
    \node[block, right of=t2, node distance=1.2cm] (s2) {$\inc{(\lift{\sigma})}$};
    \node[block, right of=s2] (p2) {$\inc{(\lift{\pi})}$};
    \node[below of=p1, node distance=.6cm] (mid) {};
    \node[block, right of=mid, node distance=.8cm] (j) {$\inc{(\lift{\bowtie})}$};
    \node[block, right of=j] (p) {$\inc{(\lift{\pi})}$};
    \node[block, right of=p] (d) {$\inc{(\lift{\distinct})}$};
    \node[right of=d, node distance=1.2cm] (V) {$\Delta$\code{V}};.8
    \draw[->] (t1) -- (s1);
    \draw[->] (s1) -- (p1);
    \draw[->] (t2) -- (s2);
    \draw[->] (s2) -- (p2);
    \draw[->] (p1) -| (j);
    \draw[->] (p2) -| (j);
    \draw[->] (j) -- (p);
    \draw[->] (p) -- (d);
    \draw[->] (d) -- (V);
\end{tikzpicture}

Use the linearity of $\sigma$ and $\pi$ to simplify this circuit (notice that
all linear operators no longer have a $\inc{\cdot}$):

\noindent
\begin{tikzpicture}[node distance=1cm,>=latex]
    \node[] (t1) {$\Delta$\code{t1}};
    \node[block, right of=t1, node distance=1cm] (s1) {$\lift{\sigma}$};
    \node[block, right of=s1] (p1) {$\lift{\pi}$};
    \node[below of=t1, node distance=1.2cm] (t2) {$\Delta$\code{t2}};
    \node[block, right of=t2, node distance=1cm] (s2) {$\lift{\sigma}$};
    \node[block, right of=s2] (p2) {$\lift{\pi}$};
    \node[below of=p1, node distance=.6cm] (mid) {};
    \node[block, right of=mid, node distance=.8cm] (j) {$\inc{(\lift{\bowtie})}$};
    \node[block, right of=j] (p) {$\lift{\pi}$};
    \node[block, right of=p, node distance=1.2cm] (d) {$\inc{(\lift{\distinct})}$};
    \node[right of=d, node distance=1.3cm] (V) {$\Delta$\code{V}};
    \draw[->] (t1) -- (s1);
    \draw[->] (s1) -- (p1);
    \draw[->] (t2) -- (s2);
    \draw[->] (s2) -- (p2);
    \draw[->] (p1) -| (j);
    \draw[->] (p2) -| (j);
    \draw[->] (j) -- (p);
    \draw[->] (p) -- (d);
    \draw[->] (d) -- (V);
\end{tikzpicture}

Finally, replace the incremental join using the formula for bilinear operators
(Theorem~\ref{bilinear}),
and the incremental $\distinct$ (Proposition~\ref{prop-inc_distinct}),
obtaining the following circuit:

\noindent
\begin{tikzpicture}[node distance=.8cm,>=latex]
    \node[] (t1) {$\Delta$\code{t1}};
    \node[block, right of=t1] (s1) {$\lift{\sigma}$};
    \node[block, right of=s1] (p1) {$\lift{\pi}$};
    \node[below of=t1, node distance=.8cm] (t2) {$\Delta$\code{t2}};
    \node[block, right of=t2] (s2) {$\lift{\sigma}$};
    \node[block, right of=s2] (p2) {$\lift{\pi}$};

    % join expansion
      \node[block, right of=p1] (jI1) {$\I$};
      \node[block, right of=p2] (jI2) {$\I$};
      \draw[->] (p1) -- (jI1);
      \draw[->] (p2) -- (jI2);
      \node[block, right of=jI2] (ZI2) {$\zm$};
      \draw[->] (jI2) -- (ZI2);
      \node[block, right of=jI1] (DI1) {$\lift\bowtie$};
      \node[block, right of=ZI2, node distance=1cm] (DI2) {$\lift\bowtie$};
      \draw[->] (jI1) -- (DI1);
      \draw[->] (ZI2) -- (DI2);
      \node[block, circle, above of=DI2, inner sep=0cm] (sum) {$+$};
      \draw[->] (DI1) -- (sum);
      \draw[->] (DI2) -- (sum);
      \draw[->] (p1) -- (DI2);
      \draw[->] (p2) -- (DI1);

    \node[block, right of=sum] (p) {$\lift{\pi}$};
    \draw[->] (sum) -- (p);
    \node[block, right of=p] (Id) {$\I$};
    \node[block, right of=Id] (zd) {$\zm$};
    \node[block, below of=zd] (H) {$\lift{H}$};
    \node[right of=H] (V) {$\Delta$\code{V}};
    \draw[->] (t1) -- (s1);
    \draw[->] (s1) -- (p1);
    \draw[->] (t2) -- (s2);
    \draw[->] (s2) -- (p2);
    \draw[->] (p) -- node (tapp) {} (Id);
    \draw[->] (Id) -- (zd);
    \draw[->] (zd) -- (H);
    \draw[->] (tapp.center) |- (H);
    \draw[->] (H) -- (V);
\end{tikzpicture}

Notice that the resulting circuit contains three integration
operations: two from the join, and one from the $\distinct$.  It also
contains two join operators.  However, the work performed by each
operator for each new input is proportional to the size of the change.

