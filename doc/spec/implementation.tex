\section{Additional Implementation Details}\label{sec:implementation}

\subsection{Checkpoint/restore}

DDlog programs are stateful streaming systems.  Fault-tolerance and 
migration for such programs requires state migration.  We claim that it is 
sufficient to checkpoint and restore the "contents" of all $\zm$ operator
in order to migrate the state of a Ddlog computation.

\subsection{Maintaining a database}

DDlog is not a database, it is just a streaming view maintenance system.
In particular, DDlog will not maintain more state than absolutely necessary
to compute the changes to the views.  There is no way to find out whether 
a specific value exists at a specific time moment within a DDlog relation.
However, a simple extension to DDlog runtime can be made to provide a \emph{view query}
API: essentially all relations that may be queried have to be maintained 
internally in an integrated form as well.  The system can then provide
an API to enumerate or query a view about element membership between 
input updates.

\subsection{Materialized views}

An incremental view maintenance system is not a database -- it only computes changes
to views when given changes to tables.  However, it can be integrated with a 
database, by providing capabilities for \emph{querying} both tables and views.
An input table is just the integral of all the changes to the table.  This makes
possible building a system that is both stateful (like a database) and streaming
(like an incremental view maintenance system).

\subsection{Maintaining input invariants}

For relational query systems there is however an important caveat: 
the proofs about the correctness
of the $C_Q$ implementing the same semantics as $Q$ all require some 
preconditions on the circuit inputs.  In particular, the semantics of $Q$
is only defined for sets.  In order for $C_Q$ to faithfully emulate
the behavior of $Q$ we must enforce the invariant that the input
relations are in fact sets.  

However, the differential streaming version of the circuits accepts
an arbitrary stream of changes to the input relations.  Not all
such streams define input relations that are sets!  For example,
consider an input stream where the first element removes a
tuple from an input relation.  The resulting \zr does not represent
a set, and thus the proof of correctness does not hold.
This problem has been well understood in the context of the
relational algebra: it is the same as the notion of positivity from~\cite{green-tcs11}.

We propose three different solutions to this problem, in increasing degrees of complexity.

\paragraph{Assume that the environment is well-behaved}

The simplest solution is to do nothing and assume that at any point in time 
the integral of the input stream of changes $i$ is a set: $\forall t \in \N . \isset(\I(i)[t])$.
This may be a reasonable assumption if the changes come from a controlled
medium, e.g., a traditional database, where they represent legal database changes.

\paragraph{Normalize input relations to sets}

In order to enforce that the input relations are always sets it is sufficient
to apply a $\distinct$ operator after integration.  

\begin{tikzpicture}[auto,node distance=1cm,>=latex]
    \node[] (input) {$i$};
    \node[block, right of=input] (I) {$\I$};
    \node[block, right of=I, node distance=1.5cm] (distinct) {$\distinct$};
    \node[block, right of=distinct, node distance=1.5cm] (q) {$\lift{C_Q}$};
    \node[block, right of=q] (D) {$\D$};
    \node[right of=D] (output) {$o$};
    \draw[->] (input) -- (I);
    \draw[->] (I) -- (distinct);
    \draw[->] (distinct) -- (q);
    \draw[->] (q) -- (D);
    \draw[->] (D) -- (output);
\end{tikzpicture}

The semantics of the resulting circuit is identical to the semantics of $\inc{\lift{C_Q}}$
for well-behaved input streams.  For non-well behaved input streams one can give
a reasonable definition: a change is applied to the input relations, and then non-relations
are normalized into relations.  Removing a non-existent element is a no-op, and adding twice
an element is the same as adding it once.

\paragraph{Use a ``change manager''}

In this solution we interpose a separate software component between the environment
and the circuit.  Let us call this a ``change manager'' (CM).  The CM
is responsible for accepting commands from the environment that perform updates on the input
relations, validating them, and building incrementally an input change, by computing the
effect of the commands.  The CM needs to maintain enough internal state to validate all commands; this 
will most likely entail maintaining the full contents of the input tables.  Note that
the input tables can be computed as the $\I$ of all input deltas ever applied.  Once all 
commands producing a change have been accepted, the environment can apply the produced input change
atomically, and obtain from the circuit the corresponding output changes.

\begin{tikzpicture}[auto,node distance=1cm,>=latex]
    \node[minimum width=.5cm, minimum height=1cm] (env) {Env};
    \node[block, right of=env, minimum width=.5cm, minimum height=1cm, node distance=1.5cm] (tm) {CM};
    \node[block, right of=tm, node distance=1.5cm] (C) {$\inc{\lift{C_Q}}$};
    \node[right of=C] (output) {$o$};
    \draw[->] (env.30) -- (tm.150);
    \draw[<-] (env.330) -- (tm.210);
    \draw[->] (tm) -- node (i) {$i$} (C);
    \draw[->] (C) -- (output);
\end{tikzpicture}