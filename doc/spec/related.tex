\section{Related work}\label{sec:related}

\subsection{Incremental View Maintenance}

\dbsp using non-nested streams is a simplified instance of a Kahn 
network~\cite{kahn-ifip74}.  Johnson~\cite{johnson-phd83}
studies a very similar computational model without nested streams and its 
expressiveness. The implementation of such streaming models of computation and their
relationship to dataflow machines has been studied by Lee~\cite{lee-ieee95}.
Lee~\cite{lee-ifip93} also introduced streams of streams and the $\lift{\zm}$ operator.

In \secref{sec:extensions} we discuss the connections with window and stream database 
queries~\cite{arasu-tr02,aurora}.

Incremental view maintenance (e.g.~\cite{gupta-idb93}) is
surveyed in~\cite{gupta-idb95}; a large bibliography is present in~\cite{motik-ai19}. 
Its most formal aspect is propagating ``deltas'' through algebraic expressions:
$Q(R+\Delta R)=Q(R)+\Delta Q(R,\Delta R)$. This work eventually crystallized in~\cite{koch-pods16}. DBSP 
incrementalization is both more modular and more fine-grain since it deals with streams of updates. 
Both~\cite{koch-pods10} and~\cite{green-tcs11} use \zrs to uniformly model insertions/deletions.

Picallo et al.~\cite{picallo-scop19} provide a general solution to IVM for
rich languages.  \dbsp requires a group structure on the values operated on; 
this assumption has two major practical benefits: it simplifies the mathematics considerably
(e.g., Picallo uses monoid actions to model changes), and it provides a general, simple
algorithm (\ref{algorithm-inc}) for incrementalizing arbitrary programs.  The downside of 
\dbsp is that one has to find a suitable group structure (e.g., \zrs for sets) to ``embed'' 
the computation.  Picallo's notion of ``derivative'' is not unique: they need creativity to choose
the right derivative definition, we need creativity to find the right group structure.

\begin{comment}
The main problem that change structures address is that the types used in programs are not
closed under subtraction (e.g., the delta between two sets is not a set). 
Although a relational \dbsp circuit computes
only on positive \zr values, its incremental version may compute on negative 
values, but the equivalence of the two programs guarantees correctness even though the
type system of \zrs does not.  \val{Safe to delete this para}
\end{comment}

Many heuristic algorithms were published for Datalog-like languages, e.g., 
counting based approaches~\cite{Dewan-iis92,motik-aaai15} that maintain the number of derivations,
DRed~\cite{gupta-sigmod93} and its variants~\cite{Ceri-VLDB91,Wolfson-sigmod91,%
Staudt-vldb96,Kotowski-rr11,Lu-sigmod95,Apt-sigmod87}, the backward-forward algorithm and variants~\cite{motik-aaai15,
Harrison-wdd92,motik-ai19}.  \dbsp is more general than these approaches.  
Interestingly, the \zrs multiplicities in our relational implementation 
are related to the counting-number-of-derivation approaches.

\dbsp is tightly related to Differential Dataflow (DD)~\cite{mcsherry-cidr13, murray-sosp13}
and its theoretical foundations~\cite{abadi-fossacs15} (and recently~\cite{mchserry-vldb20,chothia-vldb16}).
All \dbsp operators are based on DD operators.  DD's computational model is more powerful than
\dbsp, since it allows past values in a stream to be "updated".
In contrast, our model assumes that the inputs of a computation arrive in the time order while allowing 
for nested time domains via the modular lifting transformer.  However, \dbsp can express both
incremental and non-incremental computations; in essence \dbsp is ``deconstructing'' DD into 
simple component building blocks; the core Proposition~\ref{prop-inc-properties} and
the Algorithm based on it~\ref{algorithm-inc} are new contributions.

\subsection{Stream computation models}

\cite{gammie-acs13} surveys the connection between synchronous digital circuits and functional programs.
Our circuits are nothing but higher order functions computing on streams (functions themselves).
The paper's main focus are circuits processing numeric data, whereas, taking advantage of our circuits'
ability to compute on arbitrary groups, we use circuits to implement incremental view maintenance for
relational databases.

DB Toaster~\cite{ahmad-vldb09} and its associated theory~\cite{koch-pods10, koch-pods16}
provide a formal treatment of incremental view maintenance in relational query languages.

Reconcilable differences~\cite{green-tcs11}

Provenance semi-rings~\cite{green-pods07}

Mamouras~\cite{mamouras-esop20}  

\subsection{Connection to synchronous circuits}

There is a vast literature on \defined{synchronous circuits}, which are well-defined
models for hardware circuits e.g.~\cite{gammie-acs13}.  These circuits also compute over infinite streams of values,
usually of Booleans $\stream{\B}$.  In a \defined{combinational circuit} the output
values depend only on the current input values.  These are pure lifted streaming
computations.  A \defined{sequential circuit} can have outputs that depend on past
input values.  These are always causal circuits.  Sequential synchronous circuits use
latches or flip-flops to store state; the latches are controlled by a global clock signal.
These correspond to the $\zm$ operator.  In a well-formed sequential circuit all back-edges
must go through some latch --- this corresponds to our circuit construction rule that
requires a delay element on each back-edge.

Languages such as Verilog or VHDL can be used to specify such circuits.  (However, both Verilog
and VHDL are strictly more powerful, and can express richer classes of circuits than just
synchronous sequential circuits.)

There is a rich literature on synchronous circuits, and some of these results are
directly applicable to the circuits we discuss.  Here are a few examples.

Retiming~\cite{leiserson-algorithmica91}
is an optimization that ``moves'' around delay elements while preserving the circuit
semantics.  Retiming is used traditionally to reduce the clock cycle by minimizing
the signal propagation delay between any pair of latches.  In our case it could be
used for minimizing the amount of internal circuit state.

In a synchronous circuit the \emph{state} is entirely stored in the latches.  
Saving and restoring the contents of the latches enables such circuits to take
a snapshot of their state and resume computation.\footnote{In Boolean
synchronous circuits this is achieved by connecting all latches into a \emph{scan 
chain} which can be read and written sequentially after stopping the circuit clock.}

Fault tolerance of synchronous
circuits is provided by replicating the state elements, to prevent accidental
state changes caused by e.g., cosmic rays.  We can borrow this idea for building
redundant distributed computations.

Pipelining digital circuits is an effective technique for increasing throughput
through parallelization, by inserting additional latches and allowing different
pipeline stages to compute concurrently on distinct stream values, at the expense 
of increased latency between the inputs and the corresponding outputs.

Digital circuit latches depend on a special ``reset'' signal to initialize their
state to a pre-established value; this corresponds to the special 0 value in our 
value domain. 

Our nested streams  
are related to the notion of delta-cycles in the definition of VHDL~\cite{baker-date96}.

