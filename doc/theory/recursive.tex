\section{Recursive queries in \dbsp}\label{sec:recursive}

Recursive queries are very useful in a many applications.
For example, many graph algorithms (such as graph reachability
or transitive closure) are naturally expressed using recursive queries.

We illustrate the implementation of recursive queries in \dbsp for
stratified Datalog.

\subsection{Implementing recursive queries}

For warm-up we start with a single recursive queries, and then we discuss
the case of mutually recursive queries.

\subsubsection{Recursive rules}\label{sec:recursion}

In Datalog a recursive rule appears when a relation that appears in the head of a rule
is also used in a positive term in the rule's body.  (Stratification disallows
the use of the same relation negated in the rule's body).
The Datalog semantics of recursive rules is to compute a fixedpoint.

Consider a Datalog program of the form:

\begin{lstlisting}[language=ddlog]
O(v) :- C(v).  // base case
O(v) :- ..., O(x), I(z), ... .  // recursive case
\end{lstlisting}

Note that relation \code{O} is recursively defined.  Let us assume wlog that the \code{O}
relation depends on two other relations (i.e., in the rule bodies defining \code{O} the
two other relations appear) : a ``base case'' relation \code{C} (which appears in a
non-recursive rule), and a relation \code{I} which appears in the recursive rule, but
does not itself depend on \code{O}.

To implement the computation of \code{O} as a circuit we perform the following algorithm:

\begin{algorithm}[recursive queries]\label{algorithm-rec}
\noindent
\begin{enumerate}[nosep, leftmargin=\parindent]
\item Implement the non-recursive relational query $R$ as described in
    \secref{sec:relational} and Table~\ref{tab:relational}; this produces
    an acyclic circuit whose inputs and outputs are a \zrs:
    \begin{center}
    \begin{tikzpicture}[auto,>=latex]
      \node[] (I) {\code{I}};
      \node[below of=I, node distance=.5cm] (O) {\code{O}};
      \node[block, right of=I] (R) {$R$};
      \node[right of=R] (o) {\code{O}};
      \draw[->] (I) -- (R);
      \draw[->] (O) -| (R);
      \draw[->] (R) -- (o);
    \end{tikzpicture}
    \end{center}

\item Lift this circuit to operate on streams:
    \begin{center}
    \begin{tikzpicture}[auto,>=latex]
      \node[] (I) {\code{I}};
      \node[below of=I, node distance=.5cm] (O) {\code{O}};
      \node[block, right of=I] (R) {$\lift R$};
      \node[right of=R] (o) {\code{O}};
      \draw[->] (I) -- (R);
      \draw[->] (O) -| (R);
      \draw[->] (R) -- (o);
    \end{tikzpicture}
    \end{center}
  We construct $\lift{R}$ by lifting each operator of the circuit individually
  according to Proposition~\ref{prop:distributivity}.

\item Build a cycle, connecting the output to the  corresponding
input via a delay:

 \begin{center}
\begin{tikzpicture}[auto,>=latex, node distance=.8cm]
  \node[] (I) {\code{I}};
  \node[block, right of=I] (R) {$\lift R$};
  \node[right of=R, node distance=1.5cm] (O) {\code{O}};
  \node[block, below of=R, node distance=.9cm] (z) {$\zm$};
  \draw[->] (I) -- (R);
  \draw[->] (R) -- node(o) {$o$} (O);
  \draw[->] (o) |- (z);
  \draw[->] (z) -- (R);
 \end{tikzpicture}
 \end{center}
\item ``Bracket'' the circuit in $\I$ and $\D$ nodes, and then in $\delta_0$ and $\int$:

\begin{center}
\begin{tikzpicture}[auto,>=latex, node distance=.8cm]
  \node[] (Iinput) {\code{I}};
  \node[block, right of=Iinput] (ID) {$\delta_0$};
  \node[block, right of=ID] (II) {$\I$};
  \node[block, right of=II] (f) {$\lift{R}$};
  \node[block, right of=f, node distance=1.5cm] (D) {$\D$};
  \node[block, right of=D] (S) {$\int$};
  \node[right of=S] (output)  {\code{O}};
  \draw[->] (Iinput) -- (ID);
  \draw[->] (ID) -- (II);
  \draw[->] (II) -- (f);
  \draw[->] (f) -- node (o) {$o$} (D);
  \draw[->] (D) -- (S);
  \draw[->] (S) -- (output);
  \node[block, below of=f, node distance=.9cm] (z) {$\zm$};
  \draw[->] (o) |- (z);
  \draw[->] (z) -- (f);
\end{tikzpicture}
\end{center}
\end{enumerate}
\end{algorithm}

This circuit as drawn is not a well-formed circuit.
\val{Why not well-formed? As far as I can see its semantics
follows from Corollary~\ref{feedback-semantics}.}
\mihai{The WFC rules require any circuit bracketed by $\delta_0$ -- $\int$ to have no other input or output edges.  It also
requires back-edges to go through a plus only.  It is
more strict than the stream computation rules.}
It can, however, be modified into an equivalent
well-formed circuit by adding two constant zero value streams:

\begin{center}
\begin{tikzpicture}[auto,>=latex]
  \node[] (Iinput) {\code{I}};
  \node[below of=Iinput] (Cinput) {\code{C}};
  \node[below of=Cinput] (Zinput) {$0$};
  \node[block, right of=Iinput] (IE) {$E$};
  \node[block, right of=Cinput] (CE) {$E$};
  \node[block, right of=Zinput] (ZE) {$E$};
  \node[block, shape=circle, right of=ZE, inner sep=0in] (Zplus) {$+$};
  \node[block, right of=CE] (f) {$\lift{R}$};
  \node[block, shape=circle, above of=f, inner sep=0in] (Cplus) {$+$};
  \node[block, right of=Cplus] (Cz) {$\zm$};
  \node[block, right of=Cz, node distance=1.5cm] (zero) {$\lift{\lambda x . 0}$};
  \node[block, right of=f, shape=circle, inner sep=0in] (plus) {$+$};
  \node[block, right of=plus, node distance=3cm] (X) {$X$};
  \node[right of=X] (output)  {\code{O}};
  \node[block, below of=plus] (z) {$\zm$};

  \draw[->] (Iinput) -- (IE);
  \draw[->] (Cinput) -- (CE);
  \draw[->] (Zinput) -- (ZE);
  \draw[->] (ZE) -- (Zplus);
  \draw[->] (CE) to[out=55,in=125] (plus);
  \draw[->] (plus) -- +(right:2.4cm) node (o) {} -- (X);
  \draw[->] (X) -- (output);
  \draw[->] (o.south) |- (zero);
  \draw[->] (Zplus) -- (f);
  \draw[->] (z) -- (Zplus);
  \draw[->] (Zplus) -- (f);
  \draw[->] (IE) -- (Cplus);
  \draw[->] (f) -- (plus);
  \draw[->] (zero) -- (Cz);
  \draw[->] (o) |- (z);
  \draw[->] (Cz) -- (Cplus);
  \draw[->] (Cplus) -- (f);
\end{tikzpicture}
\end{center}

\begin{theorem}
If $\isset(\code{I})$ and $\isset(\code{C})$, the output of the circuit above is
the relation $\code{O}$ as defined by the Datalog semantics of recursive relations
as a function of the input relations \code{I} and \code{C}.
\end{theorem}

\begin{proof}
The proof is by structural induction on the structure of the circuit.
As a basis for induction we assume that the circuit $R$ correctly implements
the semantics of the recursive rule body when treating \code{O} as an
independent input.  We need to prove that the output of circuit encompassing
$R$ produces the correct value of the \code{O} relation, as defined by
the recursive Datalog equation.

Let us compute the contents of the $o$ stream, produced at the output
of the $\distinct$ operator.  We will show that this stream is composed
of increasing approximations of the value of \code{O}, and in fact
$\code{O} = \lim_{t \to \infty} o[t]$ if the limit exists.

We define the following one-argument function: $R'(x) = \lambda x . R(\code{I}, x)$.
Notice that the left input of the $\lift{R}$ block is a constant stream
with the value \code{I}.  Due to the stratified nature of the language,
we must have $\ispositive(R')$, so $\forall x . R'(x) \geq x$.
Also $\lift{R'}$ is time-invariant, so $R'(0) = 0$.

With this notation for $R'$ the previous circuit has the
output as the following simpler circuit:

\begin{center}
\begin{tikzpicture}[auto,>=latex]
  \node[] (Cinput) {\code{C}};
  \node[block, right of=Cinput] (CE) {$E$};
  \node[block, below of=CE] (f) {$\lift{R'}$};
  \node[block, right of=f, shape=circle, inner sep=0in] (plus) {$+$};
  \node[block, right of=plus, node distance=1.5cm] (X) {$X$};
  \node[right of=X] (output)  {\code{O}};
  \draw[->] (Cinput) -- (CE);
  \draw[->] (f) -- (plus);
  \draw[->] (plus) -- node (o) {$o$} (X);
  \draw[->] (X) -- (output);
  \node[block, below of=plus] (z) {$\zm$};
  \draw[->] (o) |- (z);
  \draw[->] (z) -| (f);
  \draw[->] (CE) -| (plus);
\end{tikzpicture}
\end{center}

We use the following notation: $x \cup y = \distinct(x + y)$.  As discussed in Section~\ref{sec:union}
the $\cup$ operation computes the same result as set union when $x$ and $y$ are sets.
With this notation let us compute the values of the $o$ stream:

$$
\begin{aligned}
o[0] =& \code{C} + R'(0) = \code{C} \cup R'(0) = \code{C} \\
o[1] =& \code{C} + R'(o[0]) = \code{C} \cup R'(\code{C}) \\
o[t] =& \code{C} + R(o[t-1]) = \code{C} \cup R'(o[t-1]) \\
\end{aligned}
$$

Defining a new helper function $S(x) = \code{C} \cup R'(x)$, the previous system of equations becomes:

$$
\begin{aligned}
o[0] =& S(0) \\
o[1] =& S(S(0)) \\
o[t] =& S(o[t-1]) \\
\end{aligned}
$$

So, by induction $o[t] = S^t(0)$, where by $S^t$ we mean $\underbrace{S \circ S \circ \ldots \circ S}_{t}$.
$S$ is monotone because $R'$ is monotone; thus, if there is a time $k$ such that $S^k(0) = S^{k+1}(0)$, we have
$\forall j \in \N . S^{k+j}(0) = S^k(0)$.

\code{O} is computed by the $X$ operator as the limit of stream $o$:
$\code{O} = X(o) = \lim_{n \to \infty} o[n]$.  If this limit exists (i.e., a fixed-point
is reached), the circuit computes the fixed point $\fix{x}{S(x)}$.  This is exactly
the definition of the Datalog semantics of a recursive relation definition: $\code{O} =
\fix{x}{\code{C} \cup R(\code{I}, x)}$.
\end{proof}

Note that the use of unbounded domains (like integers with arithmetic operations) does
not guarantee convergence for all programs.

Our circuit implementation is in fact computing the value of relation \code{O} using the standard
\defined{na\"{\i}ve evaluation} algorithm (e.g., see Algorithm~1 from \cite{greco-sldm15}).

Observe that the ``inner'' part of the circuit is the incremental
form of another circuit, since is ``sandwiched'' between $\I$ and $\D$ operators.
According to Proposition~\ref{prop-inc-properties}, part 7, the circuit can be
rewritten as:

\begin{equation}
\begin{aligned}
  \label{eq:seminaive}
\begin{tikzpicture}[auto,>=latex]
  \node[] (Cinput) {\code{C}};
  \node[below of=Cinput] (Iinput) {\code{I}};
  \node[block, right of=Iinput] (Idelta) {$\delta_0$};
  \node[block, above of=Idelta] (Cdelta) {$\delta_0$};
  \node[block, right of=Idelta] (f) {$\inc{\lift{R}}$};
  \node[block, right of=f, shape=circle, inner sep=0in] (plus) {$+$};
  \node[block, right of=plus, node distance=1.5cm] (S) {$\int$};
  \node[right of=S] (output)  {\code{O}};
  \node[block, below of=plus] (z) {$\zm$};
  \draw[->] (Iinput) -- (Idelta);
  \draw[->] (Cinput) -- (Cdelta);
  \draw[->] (Cdelta) -| (plus);
  \draw[->] (f) -- (plus);
  \draw[->] (plus) -- node (o) {} (S);
  \draw[->] (S) -- (output);
  \draw[->] (o) |- (z);
  \draw[->] (z) -| (f);
  \draw[->] (Idelta) -- (f);
\end{tikzpicture}
\end{aligned}
\end{equation}

This form of the circuit is effectively implementing the \defined{semi-na\"{\i}ve evaluation}
of the same relation (Algorithm~2 from~\cite{greco-sldm15}).  So the correctness of semi-na\"{\i}ve evaluation
is an immediate consequence of the cycle rule from Proposition~\ref{prop-inc-properties}.

\begin{comment}
Let us notice that the combination $\int \circ \D$ applied to a
monotone stream will produce that fixed-point value of the input
stream (if it exists).  This suggests a simple practical implementation
for for $\int$ operator: stop aggregating at the first input that is 0.
For monotone loop bodies involving a positive query $Q$ this implementation is
correct.
\end{comment}

\subsection{Example: a recursive query in \dbsp}\label{sec:recursive-example}

Here we apply the algorithm~\ref{algorithm-rec}, that converts a
recursive query into a DBSP circuit, to a concrete Datalog program.
The program computes the transitive closure of a directed graph:

\begin{lstlisting}[language=ddlog,basicstyle=\small]
// Edge relation with head and tail
input relation E(h: Node, t: Node)
// Reach relation with source s and sink t
output relation R(s: Node, t: Node)
R(x, y) :- E(x, y).
R(x, y) :- E(x, z), R(z, y).
\end{lstlisting}

Step 1: we ignore the fact that R is both an input and an output and we implement
the \dbsp circuit corresponding to the body of the query.  This query could be expressed
in SQL as:

\begin{lstlisting}[language=SQL,basicstyle=\small]
( SELECT * FROM E)
UNION
( SELECT E.h , R.t
  FROM E JOIN R
  ON E.t = R.s )
\end{lstlisting}

This query generates a \dbsp circuit with inputs \code{E} and \code{R}:

\begin{center}
\begin{tikzpicture}[>=latex, node distance=1.2cm]
  \node[] (E) {\code{E}};
  \node[above of=E, node distance=.6cm] (R1) {\code{R}};
  \node[block, right of=R1] (j) {$\bowtie_{t=s}$};
  \node[block, right of=j] (pj) {$\pi_{h, t}$};
  \node[block, circle, below of=pj, inner sep=0cm, node distance=.6cm] (plus) {$+$};
  \node[block, right of=plus] (d) {$\distinct$};
  \node[right of=d] (R) {\code{R}};
  \draw[->] (R1) -- (j);
  \draw[->] (E) -- (j);
  \draw[->] (j) -- (pj);
  \draw[->] (E) -- (plus);
  \draw[->] (pj) -- (plus);
  \draw[->] (plus) -- (d);
  \draw[->] (d) -- (R);
\end{tikzpicture}
\end{center}

Step 2: Lift the circuit by lifting each operator pointwise:

\begin{center}
\begin{tikzpicture}[>=latex, node distance=1.5cm]
  \node[] (E) {\code{E}};
  \node[above of=E, node distance=.8cm] (R1) {\code{R}};
  \node[block, right of=R1] (j) {$\lift{\bowtie_{t=s}}$};
  \node[block, right of=j] (pj) {$\lift{\pi_{h, t}}$};
  \node[block, circle, below of=pj, node distance=.6cm, inner sep=0cm] (plus) {$+$};
  \node[block, right of=plus] (d) {$\lift{\distinct}$};
  \node[right of=d] (R) {\code{R}};
  \draw[->] (R1) -- (j);
  \draw[->] (E) -- (j);
  \draw[->] (j) -- (pj);
  \draw[->] (E) -- (plus);
  \draw[->] (pj) -- (plus);
  \draw[->] (plus) -- (d);
  \draw[->] (d) -- (R);
\end{tikzpicture}
\end{center}

Step 3: Connect the feedback loop implied by relation \code{R}:

\begin{center}
\begin{tikzpicture}[>=latex]
  \node[] (E) {\code{E}};
  \node[right of=E] (empty) {};
  \node[block, above of=empty, node distance=.8cm] (j) {$\lift{\bowtie_{t=s}}$};
  \node[block, right of=j, node distance=1.4cm] (pj) {$\lift{\pi_{h, t}}$};
  \node[block, circle, below of=pj, node distance=.8cm, inner sep=0cm] (plus) {$+$};
  \node[block, right of=plus, node distance=1.3cm] (d) {$\lift{\distinct}$};
  \draw[->] (E) -- (j);
  \draw[->] (j) -- (pj);
  \draw[->] (E) -- (plus);
  \draw[->] (pj) -- (plus);
  \draw[->] (plus) -- (d);

  % generic part
  \node[right of=d, node distance=1.4cm] (output)  {\code{R}};
  \draw[->] (d) -- node (o) {} (output);
  \node[block, above of=j, node distance=.8cm] (z) {$\zm$};
  \draw[->] (o.center) |- (z);
  \draw[->] (z) -- (j);
\end{tikzpicture}
\end{center}

Step 4: ``bracket'' the circuit, once with $\I$-$\D$, then with $\delta_0$-$\int$:

\noindent
\begin{equation}\label{recursive-join}
\begin{tikzpicture}[>=latex]
  \node[] (Einput) {\code{E}};
  % generic part
  \node[block, right of=Einput, node distance=.8cm] (ID) {$\delta_0$};
  \node[block, right of=ID, node distance=.8cm] (E) {$\I$};

  % relational query
  \node[block, above of=E, node distance=.8cm] (j) {$\lift{\bowtie_{t=s}}$};
  \node[block, right of=j, node distance=1.5cm] (pj) {$\lift{\pi_{h, t}}$};
  \node[block, circle, below of=pj, node distance=.8cm, inner sep=0cm] (plus) {$+$};
  \node[block, right of=plus, node distance=1.3cm] (d) {$\lift{\distinct}$};
  \draw[->] (E) -- (j);
  \draw[->] (j) -- (pj);
  \draw[->] (E) -- (plus);
  \draw[->] (pj) -- (plus);
  \draw[->] (plus) -- (d);

  % generic part
  \node[block, right of=d, node distance=1.3cm] (D) {$\D$};
  \node[block, right of=D, node distance=.8cm] (S) {$\int$};
  \node[right of=S, node distance=.8cm] (output)  {\code{R}};
  \draw[->] (Einput) -- (ID);
  \draw[->] (ID) -- (E);
  \draw[->] (d) -- node (o) {} (D);
  \draw[->] (D) -- (S);
  \draw[->] (S) -- (output);
  \node[block, above of=j, node distance=.8cm] (z) {$\zm$};
  \draw[->] (o.center) |- (z);
  \draw[->] (z) -- (j);
\end{tikzpicture}
\end{equation}

The above circuit is a complete implementation of the non-streaming
recursive query; given an input relation \code{E} it will produce
its transitive closure \code{R} at the output.

Now we use semina\"ive evaluation~\ref{eq:seminaive} to rewrite the circuit
(to save space in the figures we will omit the indices from $\pi$ and $\sigma$
in the subsequent figures):

\begin{center}
\begin{tikzpicture}[>=latex, node distance=1.4cm]
  \node[] (Einput) {\code{E}};
  % generic part
  \node[block, right of=Einput, node distance=.8cm] (E) {$\delta_0$};

  % relational query
  \node[block, above of=E, node distance=.9cm] (j) {$\inc{(\lift{\bowtie})}$};
  \node[block, right of=j, node distance=1.5cm] (pj) {$\inc{(\lift{\pi})}$};
  \node[block, circle, below of=pj, node distance=.9cm, inner sep=0cm] (plus) {$+$};
  \node[block, right of=plus, node distance=1.5cm] (d) {$\inc{(\lift{\distinct})}$};
  \draw[->] (E) -- (j);
  \draw[->] (j) -- (pj);
  \draw[->] (E) -- (plus);
  \draw[->] (pj) -- (plus);
  \draw[->] (plus) -- (d);

  % generic part
  \node[block, right of=d, node distance=1.6cm] (S) {$\int$};
  \node[right of=S, node distance=1cm] (output)  {\code{R}};
  \draw[->] (Einput) -- (E);
  \draw[->] (d) -- node (o) {} (S);
  \draw[->] (S) -- (output);
  \node[block, above of=j, node distance=.9cm] (z) {$\zm$};
  \draw[->] (o.center) |- (z);
  \draw[->] (z) -- (j);
\end{tikzpicture}
\end{center}

Using the linearity of $\lift\pi$, this can be rewritten as:

\begin{center}
\begin{tikzpicture}[>=latex, node distance=1.3cm]
  \node[] (Einput) {\code{E}};
  % generic part
  \node[block, right of=Einput, node distance=.8cm] (E) {$\delta_0$};

  % relational query
  \node[block, above of=E, node distance=.9cm] (j) {$\inc{(\lift{\bowtie})}$};
  \node[block, right of=j] (pj) {$\lift{\pi}$};
  \node[block, circle, below of=pj, node distance=.9cm, inner sep=0cm] (plus) {$+$};
  \node[block, right of=plus, node distance=1.6cm] (d) {$\inc{(\lift{\distinct})}$};
  \draw[->] (E) -- (j);
  \draw[->] (j) -- (pj);
  \draw[->] (E) -- (plus);
  \draw[->] (pj) -- (plus);
  \draw[->] (plus) -- (d);

  % generic part
  \node[block, right of=d, node distance=1.6cm] (S) {$\int$};
  \node[right of=S, node distance=.8cm] (output)  {\code{R}};
  \draw[->] (Einput) -- (E);
  \draw[->] (d) -- node (o) {} (S);
  \draw[->] (S) -- (output);
  \node[block, above of=j, node distance=.9cm] (z) {$\zm$};
  \draw[->] (o.center) |- (z);
  \draw[->] (z) -- (j);
\end{tikzpicture}
\end{center}

\subsubsection{Mutually recursive rules}\label{sec:mutually-recursive}

Given a stratified Datalog program we can compute a graph where relations are nodes and dependences
between relations are edges.  We then compute the strongly connected components of this graph.
All relations from a strongly-connected component are mutually recursive.

Let us consider the implementation of a single strongly-connected component defining $n$
relations $O_i, i \in [n]$.  We can assume wlog that the definition of $O_i$ has the following
structure:
\newcommand{\tns}{\code{:-}}
\newcommand{\cd}{\code{.}}
$$
\begin{array}{lll}
O_i(v) &\tns& C_i(v)\cd \\
O_i(v) &\tns&  I_i(x), O_0(v_0), O_1(v_1), \ldots, O_{n-1}(v_{n-1}), v = \ldots \cd \\
\end{array}
$$

There are exactly $n$ base cases, one defining each $O_i$.  Also, we assume that each $O_i$ relation
depends on an external relation $I_i$, which does not itself depend on any $O_k$.

To compile these into circuits we generalize the algorithm from Section~\ref{sec:recursion}:

\begin{enumerate}
    \item For each recursive rule for $O_i$ implement a circuit $R_i$
    that treats all $O_j . j \in [n]$ and $I_i$ in the rule body as inputs.
    Here is the circuit $R_0$ for relation $O_0$:

    \begin{tikzpicture}[auto,>=latex]
      \node[] (i) {$I_0$};
      \node[below of=i, node distance=.7cm] (o0) {$O_0$};
      \node[below of=o0, node distance=.3cm] (odots) {$\ldots$};
      \node[below of=odots, node distance=.3cm] (on) {$O_{n-1}$};
      \node[block, right of=i, node distance=2cm] (R) {$R_0$};
      \node[right of=R] (output) {};
      \draw[->] (o0) -- (R);
      \draw[->] (on) -- (R);
      \draw[->] (i) -- (R);
      \draw[->] (R) -- (output);
    \end{tikzpicture}

    \item Embed each circuit $R_i$ as part of a ``widget'' as follows:

    \begin{tikzpicture}[auto,>=latex]
      \node[] (c0) {$C_0$};
      \node[below of=c0, node distance=.7cm] (i) {$I_0$};
      \node[below of=i, node distance=.7cm] (o0) {$O_0$};
      \node[below of=o0, node distance=.3cm] (odots) {$\ldots$};
      \node[below of=odots, node distance=.3cm] (on) {$O_{n-1}$};
      \node[block, right of=i, node distance=2cm] (R) {$R_0$};
      \node[block, right of=R, shape=circle, inner sep=0in] (plus) {$+$};
      \node[right of=plus, node distance=1.5cm] (o) {$O_0'$};
      \draw[->] (o0) -- (R);
      \draw[->] (on) -- (R);
      \draw[->] (i) -- (R);
      \draw[->] (R) -- (plus);
      \draw[->] (plus) -- (o);
      \draw[->] (c0) -| (plus);
    \end{tikzpicture}

    \item Finally, lift each such widget and connect them to each other via
    $\zm$ operators to the corresponding recursive inputs.
    The following is the shape of the circuit computing $O_0$;
    the $O_j'$ sources correspond to the widget outputs of the other
    recursive circuits:

    \begin{tikzpicture}[auto,>=latex]
      \node[] (c0) {$C_0$};
      \node[block, right of=c0] (ec0) {$E$};
      \node[below of=c0, node distance=.7cm] (i) {$I_0$};
      \node[block, right of=i] (ei) {$E$};
      \node[right of=ei] (marker) {};
      \node[block, below of=marker, node distance=.7cm] (z0) {$\zm$};
      \node[below of=z0, node distance=.35cm] (dots) {$\ldots$};
      \node[block, below of=dots, node distance=.35cm] (zn) {$\zm$};
      \node[block, right of=ei, node distance=2.5cm] (R) {$\lift{R_0}$};
      \node[block, right of=R, shape=circle, inner sep=0in] (plus) {$+$};
      \node[block, right of=plus, node distance=2cm] (X) {$X$};
      \node[right, right of=X] (o) {$O_0$};
      \node[right of=zn, node distance=4cm] (on) {$O_{n-1}'$};

      \draw[->] (c0) -- (ec0);
      \draw[->] (i) -- (ei);
      \draw[->] (ei) -- (R);
      \draw[->] (R) -- (plus);
      \draw[->] (plus) -- node (oo) {$O_o'$} (X);
      \draw[->] (ec0) -| (plus);
      \draw[->] (X) -- (o);
      \draw[->] (oo) |- (z0);
      \draw[->] (on) -- (zn);
      \draw[->] (z0) -- (R);
      \draw[->] (zn) -- (R);
    \end{tikzpicture}
\end{enumerate}

\begin{theorem}
The program defined by the previous circuit computes
the relations $O_i$ as a function of the input relations $I_j$ and $C_i$.
\end{theorem}
\begin{proof}
TODO.
\end{proof}

\paragraph{Example: mutually recursive relations}

Consider the Datalog program below computing the transitive closure of a graph
having two kinds of edges, blue (B) and red (R):

\begin{lstlisting}[language=ddlog]
P(x,y) :- B(x,y).
Q(x,y) :- R(x,y).
P(x,y) :- B(x,z), Q(z,y).
Q(z,y) :- R(x,z), P(z,y).
O(x,y) :- P(x,y).
O(x,y) :- Q(x,y).
\end{lstlisting}

The program defined by the following circuit computes
the relation $\code{O}$ as a function of the input relations $\code{R}, \code{B}$:

\val{I will add in Section~\ref{sec:causal} a consequence of Corollary~\ref{feedback-semantics} to justify the well-definedness of this.}

\begin{center}
\begin{tikzpicture}[auto,>=latex]
\node[] (inputB) {\code{B}};
\node[block, right of=inputB] (EB) {$E$};

\node[block, shape=circle, right of=EB, node distance=1.5cm, inner sep=0in] (plusP) {$+$};
\node[block, right of=plusP, node distance=1.5cm] (distinctP) {$\lift{\distinct}$};
\node[block, below of=distinctP] (zP) {$\zm$};
\node[block, below of=zP] (zQ) {$\zm$};
\node[block, below of=zQ] (distinctQ) {$\lift{\distinct}$};
\node[block, right of=distinctP, node distance=1.5cm] (xP) {$X$};
\node[block, right of=distinctQ, node distance=1.5cm] (xQ) {$X$};

\node[block, shape=circle, left of=distinctQ, node distance=1.5cm, inner sep=0in] (plusQ) {$+$};
\node[block, left of=zP, node distance=1.5cm] (joinP) {$\lift{\bowtie}$};
\node[block, left of=zQ, node distance=1.5cm] (joinQ) {$\lift{\bowtie}$};

\node[block, left of=plusQ, node distance=1.5cm] (ER) {$E$};
\node[left of=ER] (inputR) {\code{R}};

\path (xP) -- node[block, shape=circle, inner sep=0cm] (sum) {$+$} (xQ);
\node[right of=sum] (O) {\code{O}};
\path (inputB) -- (inputR) ;

\draw[->] (inputB) -- (EB);
\draw[->] (inputR) -- (ER);
\draw[->] (EB) -- (plusP);
\draw[->] (ER) -- (plusQ);
\draw[->] (plusP) -- (distinctP);
\draw[->] (plusQ) -- (distinctQ);
\draw[->] (zP) -- (joinQ);
\draw[->] (zQ) -- (joinP);
\draw[->] (EB) -- (joinP);
\draw[->] (ER) -- (joinQ);
\draw[->] (joinQ) -- (plusQ);
\draw[->] (joinP) -- (plusP);
\draw[->] (distinctP) -- (zP);
\draw[->] (distinctQ) -- (zQ);
\draw[->] (distinctP) -- (xP);
\draw[->] (distinctQ) -- (xQ);
\draw[->] (xP) -- node (p) {\code{P}} (sum);
\draw[->] (xQ) -- node (q) [right] {\code{Q}} (sum);
\draw[->] (sum) -- (O);
\end{tikzpicture}
\end{center}
