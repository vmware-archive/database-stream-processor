\section{Incremental computation}\label{sec:incremental}

In this section we formally define incremental computations over
streams and analyze their properties.

\begin{definition}
Given a unary stream operator $Q: \stream{A} \to \stream{B}$ we define the
\defined{incremental version} of $Q$ as $\inc{Q} \defn \D \circ Q \circ \I$.
$\inc{Q}$ has the same ``type'' as $Q$: $\inc{Q}: \stream{A} \to \stream{B}$.
For an operator with multiple inputs we define
the incremental version by applying $\I$ to each input independently:
e.g., if $T: \stream{A} \times \stream{B} \rightarrow \stream{C}$ then
$\inc{T}: \stream{A} \times \stream{B} \rightarrow \stream{C}$
and $\inc{T}(a, b) \defn \D (T(\I(a), \I(b)))$.
\end{definition}

The following diagram illustrates the intuition behind this definition:

\begin{center}
\begin{tikzpicture}[auto,>=latex]
    \node[] (input) {$\Delta s$};
    \node[block, right of=input] (I) {$\I$};
    \node[block, right of=I] (Q) {$Q$};
    \node[block, right of=Q] (D) {$\D$};
    \node[right of=D] (output) {$\Delta o$};
    \draw[->] (input) -- (I);
    \draw[->] (I) -- node (s) {$s$} (Q);
    \draw[->] (Q) -- node (o) {$o$} (D);
    \draw[->] (D) -- (output);
\end{tikzpicture}
\end{center}

If $Q(s) = o$ is a computation, then $\inc{Q}$ performs
the ``same'' computation as $Q$,
but between streams of changes $\Delta s$ and $\Delta o$.
This is the diagram from the
introduction, substituting $\Delta s$ for the transaction stream $T$,
and $o$ for the stream of view versions $V$.

Notice that our definition of incremental computation is meaningful only for \emph{streaming}
computations; this is in contrast to classic definitions, e.g.~\cite{gupta-idb95} which
consider only one change.  Generalizing the definition to operate on streams gives us
additional power, especially when operating with recursive queries.

The following proposition is one of our central results.

\begin{proposition}(Properties of the incremental version):\\
\label{prop-inc-properties}
For computations of appropriate types, the following hold:
\begin{description}
\item[inversion:] $Q\mapsto\inc{Q}$ is bijective; its inverse is $Q\mapsto \I\circ Q\circ\D$.
\item[invariance:] $\inc{+} = +, \inc{(\zm)} = \zm, \inc{-} = -, \inc{\I}=\I, \inc{\D}=\D$
\item[push/pull:]
    $Q \circ \I = \I \circ \inc{Q}$; $\D\circ Q = \inc{Q}\circ\D$
\item[chain:] $\inc{(Q_1\circ Q_2)} = \inc{Q_1}\circ\inc{Q_2}$ (This generalizes to operators with multiple inputs.)
\item[add:] $\inc{(Q_1 + Q_2)} = \inc{Q_1} + \inc{Q_2}$
\item[cycle:] $\inc{(\lambda s. \fix{\alpha}{T(s,\zm(\alpha)}))} = \lambda s. \fix{\alpha}{\inc{T}(s,\zm(\alpha)})$
\end{description}
\end{proposition}
\begin{proof}

The inversion and push-pull properties follow straightforwardly from the fact
that $\I$ and $\D$ are inverses of each other.

For proving invariance we have $\inc{+}(a, b) \defn \D(\I(a) + \I(b)) = a + b$, due to
linearity of $\I$.
$\inc{-}(a) = \D(-\I(a)) = \D(0 - \I(a)) = \D(\I(0) - \I(a)) = \D(\I(0 - a)) = -a$,
also due to linearity of $\I$.

The chain rule follows from push-pull. Indeed,
$$
\I\circ\inc{Q_1}\circ\inc{Q_2}=Q_1\circ\I\circ\inc{Q_2}=Q_1\circ Q_2\circ\I
$$

I.e., we have the following sequence of equivalent circuits:

\begin{center}
\begin{tabular}{m{7.5cm}m{1cm}}
\begin{tikzpicture}[auto,>=latex]
  \node[] (input) {$i$};
  \node[block, right of=input] (I) {$\I$};
  \node[block, right of=I] (Q1) {$Q_1$};
  \node[block, right of=Q1] (Q2) {$Q_2$};
  \node[block, right of=Q2] (D) {$\D$};
  \node[right of=D] (output)  {$o$};
  \draw[->] (input) -- (I);
  \draw[->] (I) -- (Q1);
  \draw[->] (Q1) -- (Q2);
  \draw[->] (Q2) -- (D);
  \draw[->] (D) -- (output);
\end{tikzpicture}
& $\cong$ \\
\begin{tikzpicture}[auto,>=latex]
  \node[] (input) {$i$};
  \node[block, right of=input] (I) {$\I$};
  \node[block, right of=I] (Q1) {$Q_1$};
  \node[block, right of=Q1] (D1) {$\D$};
  \node[block, right of=D1] (I1) {$\I$};
  \node[block, right of=I1] (Q2) {$Q_2$};
  \node[block, right of=Q2] (D) {$\D$};
  \node[right of=D] (output)  {$o$};
  \draw[->] (input) -- (I);
  \draw[->] (I) -- (Q1);
  \draw[->] (Q1) -- (D1);
  \draw[->] (D1) -- (I1);
  \draw[->] (I1) -- (Q2);
  \draw[->] (Q2) -- (D);
  \draw[->] (D) -- (output);
\end{tikzpicture} & $\cong$ \\
%\noindent which, due to associativity of function composition, is the same as:
\begin{tikzpicture}[auto,>=latex]
  \node[] (input) {$i$};
  \node[block, right of=input] (Q1) {$\inc{Q_1}$};
  \node[block, right of=Q1, node distance=1.5cm] (Q2) {$\inc{Q_2}$};
  \node[right of=Q2] (output)  {$o$};
  \draw[->] (input) -- (Q1);
  \draw[->] (Q1) -- (Q2);
  \draw[->] (Q2) -- (output);
\end{tikzpicture}
\end{tabular}
\end{center}

Here is a version of the chain rule with a binary operator:

\begin{center}
\begin{tabular}{m{7.5cm}m{1cm}}
\begin{tikzpicture}[auto,node distance=1cm,>=latex]
    \node[] (a) {$a$};
    \node[block, right of=a] (ai) {$\I$};
    \node[below of=a] (b) {$b$};
    \node[block, right of=b] (bi) {$\I$};
    \node[block, right of=ai] (q1) {$Q_1$};
    \node[below of=q1, node distance=.5cm] (midway) {};
    \node[block, right of=bi] (q2) {$Q_2$};
    \node[block, right of=midway] (q) {$T$};
    \node[block, right of=q] (D) {$\D$};
    \node[right of=D] (output) {$o$};
    \draw[->] (a) -- (ai);
    \draw[->] (b) -- (bi);
    \draw[->] (ai) -- (q1);
    \draw[->] (bi) -- (q2);
    \draw[->] (q1) -| (q);
    \draw[->] (q2) -| (q);
    \draw[->] (q) -- (D);
    \draw[->] (D) -- (output);
\end{tikzpicture}
& $\cong$ \\
\begin{tikzpicture}[auto,node distance=1cm,>=latex]
    \node[] (a) {$a$};
    \node[block, right of=a] (ai) {$\I$};
    \node[block, right of=ai] (q1) {$Q_1$};
    \node[block, right of=q1] (d1) {$\D$};
    \node[block, right of=d1] (i1) {$\I$};

    \node[below of=a] (b) {$b$};
    \node[block, right of=b] (bi) {$\I$};
    \node[block, right of=bi] (q2) {$Q_2$};
    \node[block, right of=q2] (d2) {$\D$};
    \node[block, right of=d2] (i2) {$\I$};

    \node[below of=i1, node distance=.5cm] (midway) {};
    \node[block, right of=midway] (q) {$T$};
    \node[block, right of=q] (D) {$\D$};
    \node[right of=D] (output) {$o$};
    \draw[->] (a) -- (ai);
    \draw[->] (ai) -- (q1);
    \draw[->] (q1) -- (d1);
    \draw[->] (d1) -- (i1);

    \draw[->] (b) -- (bi);
    \draw[->] (bi) -- (q2);
    \draw[->] (q2) -- (d2);
    \draw[->] (d2) -- (i2);

    \draw[->] (i1) -| (q);
    \draw[->] (i2) -| (q);
    \draw[->] (q) -- (D);
    \draw[->] (D) -- (output);
\end{tikzpicture} & $\cong$ \\
\begin{tikzpicture}[auto,node distance=1cm,>=latex]
    \node[] (a) {$a$};
    \node[below of=a] (b) {$b$};
    \node[block, right of=a] (q1) {$\inc{Q_1}$};
    \node[below of=q1, node distance=.5cm] (midway) {};
    \node[block, right of=b] (q2) {$\inc{Q_2}$};
    \node[block, right of=midway] (q) {$\inc{T}$};
    \node[right of=q] (output) {$o$};
    \draw[->] (a) -- (q1);
    \draw[->] (b) -- (q2);
    \draw[->] (q1) -| (q);
    \draw[->] (q2) -| (q);
    \draw[->] (q) -- (output);
\end{tikzpicture}
\end{tabular}
\end{center}

The add rule follows from push/pull and the linearity of $\I$ (or $\D$). Indeed,
$$
\I\circ(\inc{Q_1}+\inc{Q_2})=\I\circ\inc{Q_1}+\I\circ\inc{Q_2}=
Q_1\circ\I+ Q_2\circ\I = (Q_1+Q_2)\circ\I
$$

I.e., the following diagrams are equivalent:

\begin{center}
\begin{tabular}{m{6cm}m{1cm}}
\begin{tikzpicture}[auto,>=latex]
  \node[] (input) {$i$};
  \node[block, right of=input] (I) {$\I$};
  \node[block, above right=0cm and 1cm of I] (Q1) {$Q_1$};
  \node[block, below right=0cm and 1cm of I] (Q2) {$Q_2$};
  \node[block, shape=circle, right of=I, node distance=2.5cm, inner sep=0pt] (plus) {$+$};
  \node[block, right of=plus,node distance=1cm] (D) {$\D$};
  \node[right of=D] (output) {$o$};
  \draw[->] (input) -- (I);
  \draw[<-] (Q1.west) -- ++(-5mm,0) |- (I);
  \draw[<-] (Q2.west) -- ++(-5mm,0) |- (I);
  \draw[->] (Q1) -| (plus);
  \draw[->] (Q2) -| (plus);
  \draw[->] (plus) -- (D);
  \draw[->] (D) -- (output);
\end{tikzpicture}
& $\cong$ \\
\begin{tikzpicture}[auto,>=latex]
  \node[] (input) {$i$};
  \node[block, above right=0cm and 1cm of input] (I1) {$\I$};
  \node[block, below right=0cm and 1cm of input] (I2) {$\I$};
  \node[block, right of=I1] (Q1) {$Q_1$};
  \node[block, right of=I2] (Q2) {$Q_2$};
  \node[block, right of=Q1] (D1) {$\D$};
  \node[block, right of=Q2] (D2) {$\D$};
  \node[block, shape=circle, right of=input, node distance=4cm, inner sep=0pt] (plus) {$+$};
  \node[right of=plus] (output) {$o$};
  \draw[<-] (I1.west) -- ++(-5mm,0) |- (input);
  \draw[<-] (I2.west) -- ++(-5mm,0) |- (input);
  \draw[->] (I1) -- (Q1);
  \draw[->] (I2) -- (Q2);
  \draw[->] (Q1) -- (D1);
  \draw[->] (Q2) -- (D2);
  \draw[->] (D1) -| (plus);
  \draw[->] (D2) -| (plus);
  \draw[->] (plus) -- (output);
\end{tikzpicture} & $\cong$ \\
\begin{tikzpicture}[auto,>=latex]
  \node[] (input) {$i$};
  \node[block, above right=0cm and 1cm of input] (Q1) {$\inc{Q_1}$};
  \node[block, below right=0cm and 1cm of input] (Q2) {$\inc{Q_2}$};
  \node[block, shape=circle, right of=input, node distance=2.5cm, inner sep=0pt] (plus) {$+$};
  \node[right of=plus] (output) {$o$};
  \draw[<-] (Q1.west) -- ++(-5mm,0) |- (input);
  \draw[<-] (Q2.west) -- ++(-5mm,0) |- (input);
  \draw[->] (Q1) -| (plus);
  \draw[->] (Q2) -| (plus);
  \draw[->] (plus) -- (output);
\end{tikzpicture}
\end{tabular}
\end{center}

The cycle rule is most interesting. First, observe that if $T$ is causal then so is
$\inc{T}$ thus both sides of the
equality are well-defined. Next, we can use again push/pull to show the equality
if we can check that
$$
\I\circ(\lambda s. \fix{\alpha}{\inc{T}(s,\zm(\alpha)}) =
(\lambda s. \fix{\alpha}{T(s,\zm(\alpha)})\circ\I
$$
that is, for any $s$,
$$
\I(\fix{\alpha}{\inc{T}(s,\zm(\alpha)}) =
\fix{\alpha}{T(\I(s),\zm(\alpha)})
$$
This follows from the following lemma.
\end{proof}

\begin{lemma}
\label{lemma-delta-fix}
If the parameters $a$ and $b$ are related by $b=\D(a)$ (equivalently $a=\I(b)$)
then the unique solutions of the fixed point equations
$$
\alpha~=~ T(a,\zm(\alpha))~~~~\mbox{and}~~~~
\beta~=~\inc{T}(b,\zm(\beta))
$$
are related by $\alpha=\I(\beta)$ (equivalently $\beta=\D(\alpha)$).
\end{lemma}
\begin{proof} (Of Lemma~\ref{lemma-delta-fix})
Let $\beta$ be the unique solution of $\beta~=~\inc{T}(\D(a),\zm(\beta))$. We verify that
$\alpha=\I(\beta)$ satisfies the equation $\alpha~=~ T(a,\zm(\alpha))$. Indeed, using the fact that $\I$ and $\D$ are inverses as well as the time-invariance of $\I$ we have
\begin{eqnarray*}
\I(\beta) & = & \I(\inc{T}(\D(a),\zm(\beta)))\\
          & = & \I(\D(T(\I(\D(a)),\I(\zm(\beta))))\\
          & = & T(a,\I(\zm(\beta))\\
          & = & T(a,\zm(\I(\beta))
\end{eqnarray*}

I.e., starting from this diagram we apply a sequence of term-rewriting se\-man\-tics-preserving
transformations:

\begin{center}
\begin{tabular}{m{5.5cm}m{1cm}}
\begin{tikzpicture}[>=latex]
    \node[] (input) {$s$};
    \node[block, right of=input] (I) {$\I$};
    \node[block, right of=I] (f) {$T$};
    \node[block, right of=f, node distance=1.5cm] (D) {$\D$};
    \node[right of=D] (output) {$o$};
    \node[block, below of=f] (z) {$\zm$};
    \draw[->] (input) -- (I);
    \draw[->] (I) -- (f);
    \draw[->] (f) -- node (mid) {} (D);
    \draw[->] (mid.center) |-  (z);
    \draw[->] (z.west) -- ++(-.3,0) |- ([yshift=1mm]f.south west);
    \draw[->] (D) -- (output);
\end{tikzpicture} & $\cong$ \\
\begin{tikzpicture}[>=latex]
    \node[] (input) {$s$};
    \node[block, right of=input] (I) {$\I$};
    \node[block, below of=I] (D0) {$\D$};
    \node[block, right of=D0] (I0) {$\I$};
    \node[above of=I0, node distance=.5cm] (midway) {};

    \node[block, right of=midway] (f) {$T$};
    \node[block, right of=f] (D) {$\D$};
    \node[right of=D] (output) {$o$};
    \node[block, below of=f, node distance=1.5cm] (I1) {$\I$};
    \node[block, left of=I1] (z) {$\zm$};
    \draw[->] (input) -- (I);
    \draw[->] (I) -| (f);
    \draw[->] (D0) -- (I0);
    \draw[->] (I0) -| (f);
    \draw[->] (f) -- (D);
    \draw[->] (D) -- node (mid) {} (output);
    \draw[->] (mid.center) |-  (I1);
    \draw[->] (I1) -- (z);
    \draw[->] (z.west) -- ++(-1.5,0) |- (D0);
\end{tikzpicture} & $\cong$ \\
\begin{tikzpicture}[>=latex]
    \node[] (input) {$s$};
    \node[block, right of=input] (f) {$\inc{T}$};
    \node[right of=f, node distance=1.5cm] (output) {$o$};
    \node[block, below of=f] (z) {$\zm$};
    \draw[->] (input) -- (f);
    \draw[->] (f) -- node (mid) {} (output);
    \draw[->] (mid.center) |-  (z);
    \draw[->] (z.west) -- ++(-.3,0) |- ([yshift=1mm]f.south west);
\end{tikzpicture}
\end{tabular}
\end{center}

\end{proof}

If we specialize the above formula for the case $T(a,b) = Q(a+b)$
(for some time-invariant operator $Q$), by us the linearity of $\I$
we get that:

\noindent
\begin{tabular}{m{6cm}m{0.5cm}m{4cm}}
\begin{tikzpicture}[auto,>=latex]
  \node[] (input) {$i$};
  \node[block, right of=input] (I0) {$\I$};
  \node[block, shape=circle, right of=I0, inner sep=0in] (plus) {$+$};
  \node[block, right of=plus] (q) {$Q$};
  \node[block, right of=q, node distance=1.5cm] (D) {$\D$};
  \node[right of=D] (output)  {$o$};
  \draw[->] (input) -- (I0);
  \draw[->] (I0) -- (plus);
  \draw[->] (plus) -- (q);
  \draw[->] (q) -- node (o) {} (D);
  \draw[->] (D) -- (output);
  \node[block, below of=q] (z) {$\zm$};
  \draw[->] (o) |- (z);
  \draw[->] (z) -| (plus);
\end{tikzpicture} & $\cong$ &
\begin{tikzpicture}[auto,>=latex]
  \node[] (input) {$i$};
  \node[block, shape=circle, right of=input, inner sep=0in] (plus) {$+$};
  \node[block, right of=plus] (q) {$\inc{Q}$};
  \node[right of=q, node distance=1.5cm] (output)  {$o$};
  \draw[->] (input) -- (plus);
  \draw[->] (plus) -- (q);
  \draw[->] (q) -- node (o) {} (output);
  \node[block, below of=q] (z) {$\zm$};
  \draw[->] (o) |- (z);
  \draw[->] (z) -| (plus);
\end{tikzpicture}
\end{tabular}

\begin{comment}
& \mbox{Initial form} \\
\begin{tikzpicture}[auto,>=latex]
  \node[] (input) {$i$};
  \node[block, right of=input] (D0) {$\D$};
  \node[block, right of=D0] (I0) {$\I$};
  \node[block, shape=circle, right of=I0, inner sep=0in] (plus) {$+$};
  \node[block, right of=plus] (q) {$Q$};
  \node[right of=q, node distance=4cm] (output)  {$o$};
  \draw[->] (input) -- (D0);
  \draw[->] (D0) -- (I0);
  \draw[->] (I0) -- (plus);
  \draw[->] (plus) -- (q);
  \draw[->] (q) -- node [pos=0.8] (o) {} (output);
  \node[block, below of=q] (I1) {$\I$};
  \node[block, right of=I1] (D1) {$\D$};
  \node[block, right of=D1] (z) {$\zm$};
  \draw[->] (o) |- (z);
  \draw[->] (z) -- (D1);
  \draw[->] (D1) -- (I1);
  \draw[->] (I1) -| (plus);
\end{tikzpicture} & \I \circ \D = \id \\
\begin{tikzpicture}[auto,>=latex]
  \node[] (input) {$i$};
  \node[block, right of=input] (D0) {$\D$};
  \node[block, shape=circle, right of=D0, inner sep=0in] (plus) {$+$};
  \node[block, right of=plus] (I) {$\I$};
  \node[block, right of=I] (q) {$Q$};
  \node[right of=q, node distance=2cm] (output)  {$o$};
  \draw[->] (input) -- (D0);
  \draw[->] (D0) -- (plus);
  \draw[->] (plus) -- (I);
  \draw[->] (I) -- (q);
  \draw[->] (q) -- node (o) {} (output);
  \node[block, below of=I] (z) {$\zm$};
  \node[block, right of=z] (D1) {$\D$};
  \draw[->] (o) |- (D1);
  \draw[->] (D1) -- (z);
  \draw[->] (z) -| (plus);but not
\end{tikzpicture} & \mbox{Linearity of }\I  \\
\begin{tikzpicture}[auto,>=latex]
  \node[] (input) {$i$};
  \node[block, right of=input] (D0) {$\D$};
  \node[block, shape=circle, right of=D0, inner sep=0in] (plus) {$+$};
  \node[block, right of=plus] (I) {$\I$};
  \node[block, right of=I] (q) {$Q$};
  \node[block, right of=q] (D1) {$\D$};
  \node[block, right of=D1] (I2) {$\I$};
  \node[right of=I2, node distance=2cm] (output)  {$o$};
  \draw[->] (input) -- (D0);
  \draw[->] (D0) -- (plus);
  \draw[->] (plus) -- (I);
  \draw[->] (I) -- (q);
  \draw[->] (q) -- (D1);
  \draw[->] (D1) -- (I2);
  \draw[->] (I2) -- node (o) {} (output);
  \node[block, below of=I] (z) {$\zm$};
  \node[block, right of=z] (D1) {$\D$};
  \draw[->] (o) |- (D1);
  \draw[->] (D1) -- (z);
  \draw[->] (z) -| (plus);
\end{tikzpicture} & \D \circ \I = \id \\
\end{comment}


%\subsection{Linear operators}

\begin{theorem}[Linear]\label{linear}
For any LTI operator $Q$ we have $\inc{Q}=Q$.
\end{theorem}

\begin{proof}
By the push/pull rule from Proposition~\ref{prop-inc-properties}
it suffices to show that $Q$ commutes with differentiation:
$$
\begin{aligned}
  \D(Q(s)) &= Q(s)-\zm(Q(s)) & \mbox{by definition of }\D \\
  &= Q(s)-Q(\zm(s)) & \mbox{by time-invariance of }Q \\
  &= Q(s-\zm(s)) & \mbox{by linearity of }Q \\
  &= Q(\D(s)) & \mbox{by definition of }\D.
\end{aligned}
$$
\end{proof}


As we have shown, the incremental version of a linear unary operator equals the operator itself.
However, this is not true, in general, for multilinear operators. Nonetheless, there is a useful relationship
between the incremental version of a multilinear operator and the operator itself. We illustrate with bilinear
operators.
\begin{comment}
\val{Such as join. from incremental maintenance literature recall the definition of "delta" for join:
$\Delta(R\bowtie S) = R\bowtie(\Delta S) \cup (\Delta R)\bowtie S \cup (\Delta R)\bowtie(\Delta S)$.}
\mihai{Indeed, the join is the main application.  Should we hint at this?  It's the first
time the relational algebra will appear in this text.}
\end{comment}

\begin{theorem}[Bilinear]\label{bilinear}
For any bilinear time-invariant operator $\times$ we have
$\inc{(a \times b)} ~=~ a \times b ~+~ \I(\zm(a)) \times b ~+~ a \times \I(\zm(b))$.
\end{theorem}

By rewriting this statement using $\Delta a$ for the stream of changes to $a$ we
get the familiar formula for incremental joins:
$\Delta(a\times b) =\Delta a \times \Delta b + a\times(\Delta b) + (\Delta a)\times b$.

In other words, the following three diagrams are equivalent:

\noindent
\begin{tabular}{m{3.0cm}m{0cm}m{3.5cm}m{0cm}m{3.5cm}}
\begin{tikzpicture}[auto,node distance=.8cm,>=latex]
    \node[] (a) {$a$};
    \node[block, right of=a] (ai) {$\I$};
    \node[below of=a, node distance=.8cm] (midway) {};
    \node[below of=midway, node distance=.8cm] (b) {$b$};
    \node[block, right of=b] (bi) {$\I$};
    \node[block, right of=midway, node distance=1.4cm] (q) {$\times$};
    \node[block, right of=q] (D) {$\D$};
    \node[right of=D] (output) {$o$};
    \draw[->] (a) -- (ai);
    \draw[->] (b) -- (bi);
    \draw[->] (ai) -| (q);
    \draw[->] (bi) -| (q);
    \draw[->] (q) -- (D);
    \draw[->] (D) -- (output);
\end{tikzpicture} &
$\cong$ &
\begin{tikzpicture}[auto,>=latex]
  \node[] (a) {$a$};
  \node[below of=input1, node distance=2cm] (b) {$b$};
  \node[block, right of=a] (I1) {$\I$};
  \node[block, below of=I1] (ab) {$\times$};
  \node[block, right of=b] (I2) {$\I$};
  \draw[->] (a) -- (I1);
  \draw[->] (b) -- (I2);
  \draw[->] (a) |- ([yshift=-1mm]ab.north west);
  \draw[->] (b) |- ([yshift=1mm]ab.south west);
  \node[block, right of=I1] (ZI1) {$\zm$};
  \node[block, right of=I2] (ZI2) {$\zm$};
  \draw[->] (I1) -- (ZI1);
  \draw[->] (I2) -- (ZI2);
  \node[block, right of=ZI1] (DI1) {$\times$};
  \node[block, right of=ZI2] (DI2) {$\times$};
  \draw[->] (ZI1) -- (DI1);
  \draw[->] (ZI2) -- (DI2);
  \node[block, circle, right of=ab, inner sep=0cm, node distance=2cm] (sum) {$+$};
  \draw[->] (ab) -- (sum);
  \draw[->] (DI1) -- (sum);
  \draw[->] (DI2) -- (sum);
  \node[right of=sum, node distance=.7cm] (output) {$o$};
  \draw[->] (sum) -- (output);
  \draw[->] (a) -- (DI2);
  \draw[->] (b) -- (DI1);
\end{tikzpicture} &
$\cong$ &
\begin{tikzpicture}[auto,>=latex,node distance=.9cm]
  \node[] (input1) {$a$};
  \node[below of=input1, node distance=1cm] (input2) {$b$};
  \node[block, right of=input1, node distance=.5cm] (I1) {$\I$};
  \node[block, right of=input2, node distance=.5cm] (I2) {$\I$};
  \draw[->] (input1) -- (I1);
  \draw[->] (input2) -- (I2);
  \node[block, right of=I2] (ZI2) {$\zm$};
  \draw[->] (I2) -- (ZI2);
  \node[block, right of=I1] (DI1) {$\times$};
  \node[block, right of=ZI2] (DI2) {$\times$};
  \draw[->] (I1) -- (DI1);
  \draw[->] (ZI2) -- (DI2);
  \node[block, circle, above of=DI2, inner sep=0cm, node distance=.5cm] (sum) {$+$};
  \draw[->] (DI1) -- (sum);
  \draw[->] (DI2) -- (sum);
  \node[right of=sum, node distance=.7cm] (output) {$o$};
  \draw[->] (sum) -- (output);
  \draw[->] (input1) -- (DI2);
  \draw[->] (input2) -- (DI1);
\end{tikzpicture}
\end{tabular}
\begin{proof}
$$
\begin{aligned}
\inc{(a \times b)} &= \D(\I(a) \times \I(b)) & \mbox{def of } \inc{\cdot} \\
             &= (\I(a) \times \I(b)) ~-~  \zm(\I(a) \times \I(b)) & \mbox{def of } \D \\
             &= \I(a) \times \I(b) ~-~  \zm(\I(a)) \times \zm(\I(b)) & \times \mbox{ time inv.}\\
             &= (a + \zm(\I(a))) \times (b + \zm(\I(b))) ~-~ \zm(\I(a)) \times \zm(\I(b)) & \mbox { $\I$ fixpoint equation } \\
             &= a \times b + \zm(\I(a)) \times b + a \times \zm(\I(b)) + \zm(\I(a)) \times \zm(\I(b)) & \\
             &  ~~~~~~~~~~~~~~-~ \zm(\I(a)) \times \zm(\I(b)) & \mbox{bilinearity} \\
             &= a \times b ~+~ \zm(\I(a)) \times b ~+~ a \times \zm(\I(b)) & \mbox{cancel} \\
             &= a \times b ~+~ \I(\zm(a)) \times b ~+~ a \times \I(\zm(b)) & \mbox{$I$ time inv.} \\
             %&= \I(a) \times \I(b) ~-~  \zm(\I(a)) \times \I(b) & \mbox{ add and sub a term} \\
             %& ~~~~~~~~~~~~+~ \zm(\I(a)) \times \I(b) ~-~ \zm(\I(a)) \times \zm(\I(b))) & \\
             %&= \D(\I(a)) \times \I(b) ~+~ \zm(\I(a)) \times \D(\I(b))) & \mbox{bilinearity, def of }\D \\
             %&= a \times \I(b) ~+~ \zm(\I(a)) \times b & \D \circ \I = \id \\
             %&= a \times \I(b) ~-~ a \times \zm(\I(b)) & \mbox{ add and sub a term} \\
             %&     ~~~~~~~~~~~~~~+~ a \times \zm(\I(b)) ~+~ \zm(\I(a)) \times b & \\
             %&= a \times \D(\I(b)) ~+~ a \times \zm(\I(b)) ~+~ \zm(\I(a)) \times b & \mbox{bilinearity, def of }\D  \\
             %&= a \times b  ~+~ \zm(\I(a)) \times b ~+~ a \times \zm(\I(b))
\end{aligned}
$$
\end{proof}

\subsection{Vector representations}\label{sec:vector-picture}

Some formulas are easier to read as mathematical expressions over ``vectors''.
We will use the following representation to depict a fragment of a stream $s$:

$\strm{0}{0}$

This ``vector'' representation fixes a time $t$ and shows the integral of the stream prefix
up to $\I(s)[t-1]$ as a rectangle, and $s[t]$ as a square.  We will use a grey rectangle to
show which part of a stream participates in a computation.  For example, we have the following
notations:

$$
\begin{aligned}
\strm{0 0} &= 0 \\
\strm{0 1} &= s[t] \\
\strm{1 0} &= \I(s)[t-1] = \zm(\I(s))[t] \\
\strm{1 1} &= \I(s)[t-1] + s[t] = \I(s)[t]
\end{aligned}
$$

Using this notation, the incremental version of a function $f$ is defined
as $\inc{f}(\strm{0}{1}) = f(\strm{1}{1}) - f(\strm{1}{0})$.

Theorem~\ref{linear} states that for a linear function $f$ we have: \\
$f(\strm{1 1}) - f(\strm{1 0}) = f(\strm{0 1}).$

The statement of Theorem~\ref{bilinear}, for a bilinear stream operation
$\times$ can be written as:
$$
\begin{aligned}
\strm{1 1} \times \strm{1 1} = &
\matmult{\strm{1 0}}{\strm{1 0}} \\ + &
\matmult{\strm{1 0}}{\strm{0 1}} \\ + &
\matmult{\strm{0 1}}{\strm{1 0}} \\ + &
\matmult{\strm{0 1}}{\strm{0 1}}.
\end{aligned}
$$

By definition $\inc{\times}$ is:
$$
\begin{aligned}
\matmult{\strm{1 1}}{\strm{1 1}} - \\
\matmult{\strm{1 0}}{\strm{1 0}} = &
\matmult{\strm{1 0}}{\strm{0 1}} \\ + &
\matmult{\strm{0 1}}{\strm{1 0}} \\ + &
\matmult{\strm{0 1}}{\strm{0 1}}.
\end{aligned}
$$
