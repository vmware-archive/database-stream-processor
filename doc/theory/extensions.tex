\section{Additional query languages}\label{sec:extensions}

In this section we describe several query models that go behind stratified Datalog
and show how they can be implemented in \dbsp.

\subsection{Aggregation}\label{sec:aggregation}

Aggregation in SQL applies a function $a$ to a whole set producing a ``scalar''
result with some type $R$: $a: 2^A \to R$.  We convert such aggregation
functions to operate on \zrs, so in \dbsp an aggregation function has
a signature $a: \Z[A] \to R$.  Correctness of the implementation is 
defined as in \refsec{sec:correctness}.

The SQL \texttt{COUNT} aggregation function is implemented on \zrs by $a_\texttt{COUNT} : \Z[A] \to \Z$, which
computes a \emph{sum} of all the element weights: $a_\texttt{COUNT}(s) = \sum_{x \in s} s[x]$.
The SQL \texttt{SUM} aggregation function is implemented on \zrs by $a_\texttt{SUM}: \Z[\R] \to \R$ which
performs a \emph{weighted sum} of all (real) values: $a_\texttt{SUM}(s) = \sum_{x \in s} x \times s[x]$.

With this definition the aggregation functions $a_\texttt{COUNT}$ and $a_\texttt{SUM}$ are in
fact linear transformations between the group $\Z[A]$ and the result group ($\Z$, and $\R$ respectively).

If the output of the \dbsp circuit can be such a ``scalar'' value, then aggregation
with a linear function is simply function application, and thus it is automatically incremental.  However, in general, for composing multiple queries
we require the result of an aggregation to be a singleton \zr (containing a single value),
and not a scalar value.  In this case the aggregation function is implemented in
\dbsp as the composition of the actual aggregation and the 
$\makeset: A \to \Z[A]$ function, 
which converts a scalar value of type $A$ to a singleton \zr, defined as follows:
$\makeset(x) \defn 1 \cdot x$.

In conclusion, the following SQL query: 
\code{SELECT SUM(c) FROM I} 
is implemented as the following circuit:

\begin{tikzpicture}[auto,>=latex]
  \node[] (I) {\code{I}};
  \node[block, right of=I] (pi) {$\pi_\texttt{C}$};
  \node[block, right of=pi] (a) {$a_\texttt{SUM}$};
  \draw[->] (I) -- (pi);
  \draw[->] (pi) -- (a);
  \node[block, right of=a, node distance=1.5cm] (m) {$\makeset$};
  \node[right of=m, node distance=1.4cm] (O) {\code{O}};
  \draw[->] (a) -- (m);
  \draw[->] (m) -- (O); 
\end{tikzpicture}

The lifted incremental version of this circuit is interesting: since $\pi$ 
and $a_\texttt{SUM}$ are linear, they are equivalent to their own incremental 
versions.  Although $\inc{(\lift \makeset)} = \D \circ \lift{\makeset} \circ \I$
cannot be simplified, it is nevertheless efficient, doing only O(1) work per
invocation, since its input and output are singleton values.

An aggregation function such as \texttt{AVG} can be written as the composition of 
a more complex linear function that computes a pair of values using 
\texttt{SUM} and \texttt{COUNT}, followed by a $\mbox{makeset}$ and a selection operation
that divides the two columns.

\begin{lstlisting}[language=SQL]
SELECT AVG(c) FROM I 
\end{lstlisting}

\begin{tikzpicture}[>=latex]
  \node[] (I) {\code{I}};
  \node[block, right of=I] (pi) {$\pi_\texttt{C}$};
  \node[block, right of=pi, node distance=1.8cm] (sc) {$(a_\texttt{SUM}, a_\texttt{COUNT})$};
  \draw[->] (I) -- (pi);
  \draw[->] (pi) -- (sc);
  \node[block, right of=sc, node distance=2cm] (m) {$\makeset$};
  \node[block, right of=m, node distance=1.4cm] (div) {$\sigma_/$};
  \node[right of=div] (O) {\code{O}};
  \draw[->] (sc) -- (m);
  \draw[->] (m) -- (div);
  \draw[->] (div) -- (O);
\end{tikzpicture}

Finally, some aggregate functions, such as \code{MIN}, are 
\emph{not} incremental in general, since for handling deletions
they may need to know the full set, and not just its changes.  The lifted
incremental version of such aggregate functions is implemented essentially
by ``brute force'', using the formula $\inc{(\lift a_\texttt{MIN})}
= \D \circ \lift{a_\texttt{MIN}} \circ \I$.  Such functions perform work
proportional to $R(s)$ at each invocation.

Note that the SQL \code{ORDER BY} directive can be modeled as
a non-linear aggregate function that emits a list.  However, such an implementation it is not efficiently incrementalizable in \dbsp. 
We leave the efficient handling of ORDER BY to future work.

Even when aggregation results do not form a group, they usually form
a structure with a zero element.  We expect that a well-defined
aggregation function maps empty \zrs to zeros in the target domain.


\subsection{Nested relations}

\subsubsection{Indexed partitions}

Let $A[K]$ be the set of functions with finite support from $K$ to $A$.
Consider a group $A$, an arbitrary set of \defined{key values} $K$, and a 
\emph{partitioning function} $k: A \to A[K]$ with the property that 
$\forall a \in A . a = \sum k(a)$.  We call elements of $A[K]$ \emph{indexed}
values of $A$ --- indexed by a key value.

Notice that $A[K]$ also has a group structure, and $k$ itself 
is a linear function (homomorphism).  As an example,
if $A = \Z[B_0 \times B_1]$, we can use for $k$ the first projection
$k: A \to \Z[A][B_0]$, where $k(a)[b] = \sum_{t \in a, t|0 = b} a[t] \cdot t$.
In other words, $k$ projects the elements in $\Z[B_0 \times B_1]$ on 
their first component.  This enables \emph{incremental computations
on nested relations}.  This is how operators such as group-by are
implemented: the result of group-by is an indexed \zr, where each 
element is indexed by the key of the group it belongs to.  Since
indexing is linear, its incremental version is very efficient.
Notice that the structure $\Z[A][K]$ represents a form of \emph{nested relation}.

\subsection{Grouping; indexed relations}\label{sec:grouping}

Pick an arbitrary set $K$ of ``key values.''
Consider the mathematical structure of finite maps from $K$ 
to \zrs over some other domain $A$: $K \to \Z[A] = \Z[A][K]$.
We call values $i$ of this structure \defined{indexed \zrs}: for
each key $k \in K$, $i[k]$ is a \zr.  Because 
the codomain $\Z[A]$ is an abelian group, this structure is itself 
an abelian group.

We use this structure to model the SQL \texttt{GROUP BY} operator in \dbsp.  
Consider a \defined{partitioning function}
$p: A \to K$ that assigns a key to any value in $A$.  We define the grouping function
$G_p: \Z[A] \to (K \to \Z[A])$ as $G_p(a)[k] \defn \sum_{x \in a.p(x)=k}a[x] \cdot x$.
When applied to a \zr $a$ this function returns a indexed \zr, where each element 
is called a \defined{grouping}\footnote{We use
``group'' for the algebraic structure and ``grouping'' for the result of \code{GROUP BY}.}: for each key $k$ a 
grouping is a \zr containing all elements of $a$ that map to $k$ 
(as in SQL, groupings are multisets, represented by \zrs).
Consider our example \zr $R$ from \refsec{sec:relational},
and a key function $p(s)$ that returns the first letter of the string 
$s$. Then we have that $G_p(R) = \{ \code{j} \mapsto \{ \code{joe} 
\mapsto 1 \}, \code{a} \mapsto \{ \code{anne} \mapsto -1 \} \}$,
i.e., grouping with this key function produces an indexed \zr with two groupings, each 
of which contains a \zr with one element.

The grouping function $G_p$ is linear for any $p$.
It follows that the group-by implementation in DBSP is automatically
incremental: given some changes
to the input relation we can apply the partitioning function
to each change separately to compute how each grouping changes.

\subsection{\texttt{GROUP BY-AGGREGATE}}

Grouping in SQL is almost always followed by aggregation.  
Let us consider an aggregation function $a: (K \times \Z[A]) \to B$ that produces values
in some group $B$, and an indexed relation of type $\Z[A][K]$, as defined above in~\refsec{sec:grouping}.
The nested relation aggregation operator $Agg_a: \Z[A][K] \to B$ applies $a$ 
to the contents of each grouping independently and adds the results:
$Agg_a(g) \defn \sum_{k \in K} a(k, g[k])$.  To apply this
to our example, let us compute the equivalent of GROUP-BY count; we use
the following aggregation function $count: K \times \Z[A]$, $count(k, s) = 
\makeset((k, a_\texttt{COUNT}(s)))$, using the \zr counting function $a_\texttt{COUNT}$ 
from~\refsec{sec:aggregation}; the notation $(a,b)$ is a pair of values $a$ and $b$.
Then we have $Agg_{count} (G_p(R)) = \{ (\code{j}, 1) \mapsto 1, 
(\code{a}, -1) \mapsto 1 \}$.

Notice that, unlike SQL, \dbsp can express naturally computations
on indexed \zrs, they are just an instance of a group structure. 
One can even implement queries that operate on each grouping 
in an indexed \zr.  However, our definition of incremental 
computation is only concerned with incrementality in the 
\emph{outermost} structures.  We leave it to future work to
explore an appropriate definition of incremental computation that
operates on the \emph{inner} relations.

A very useful operation on nested relations is \defined{flatmap}, which is
essentially the inverse of partitioning, converting an indexed
\zr into a \zr: $\mbox{flatmap}: \Z[A][K] \to \Z[A \times K]$.
$\mbox{flatmap}$ is in fact a particular instance of aggregation,
using the aggregation function $a: K \times \Z[A] \to \Z[A \times K]$
defined by $a(k, s) = \sum_{x \in s[k]} s[k][x] \cdot (k, x).$
For our previous example, $\mbox{flatmap}(G_p(R)) = \{ (\code{j}, \code{joe}) \mapsto 1, 
(\code{a}, \code{anne}) \mapsto -1 \}$.

If we use an aggregation function $a: K \times Z[A]$ that is linear in its
second argument, then the aggregation operator $Agg_a$ is linear, and
thus fully incremental.  As a consequence, $\mbox{flatmap}$ is linear.  
However, many practical aggregation functions for nested relations are in fact 
not linear; an example is the $count$ function above, which is not linear
since it uses the $\makeset$ non-linear function.  Nevertheless, while 
the incremental evaluation of such functions is not fully incremental, 
it is at least partly incremental: when applying a change to groupings, the aggregation 
function only needs to be re-evaluated \emph{for groupings that have changed}.


\subsection{Streaming joins}

Consider a binary query $T(s, t) = \I(t)~~\lift{\bowtie}~~s$.  This is the
\emph{relation-to-stream join} operator supported by streaming databases like ksqlDB~\cite{jafarpour-edbt19}.
Stream $t$ carries changes to a relation, while $s$ carries arbitrary data, e.g., logs
or telemetry data points. $T$ ``discards'' values from $s$ after matching them against the accumulated contents of the relation.

\begin{center}
\noindent
\begin{tikzpicture}[>=latex]
  \node[] (t) {$t$};
  \node[below of=t] (s) {$s$};
  \node[right of=t, block] (I) {$\I$};
  \node[below of=t, node distance=.5cm] (mid) {};
  \node[block, right of=mid, node distance=2cm] (j) {$\bowtie$};
  \node[right of=j] (o) {$o$};
  \draw[->] (s) -| (j);
  \draw[->] (t) -- (I);
  \draw[->] (I) -| (j);
  \draw[->] (j) -- (o);
\end{tikzpicture}
\end{center}

\subsection{Explicit delay}

So far the $\zm$ operator was confined to its implicit use in integration or
differentiation.  However, it can be exposed as a primitive operation that
can be applied to streams or collections.  This enables programs that can
perform time-based window computations over streams, and convolution-like
operators.  


\subsection{Multisets/bags}

Since \zrs generalize multisets and bags, it is easy to implement query
operators that compute on such structures.  For example, while SQL \code{UNION}
is \zr addition followed by $\distinct$, \code{UNION ALL} is just \zr addition.


\subsection{Window aggregates}

Streaming databases often organize the contents of streams into windows, 
which store a subset of data points with a predefined range of timestamps.
The circuit below (a convolution filter in DSP) computes a \emph{fixed-size sliding-window aggregate}
over the last four timestamps defined by the $T_i$ functions.

\begin{center}
\begin{tikzpicture}[>=latex]
    \node[] (input) {$s$};
    \node[block, right of=input, node distance=1.5cm] (f0) {$T_0$};
    \node[below of=input, node distance=1cm] (fake) {};
    \node[block, right of=fake, node distance=1cm] (z0) {$\zm$};
    \node[right of=input, node distance=.35cm] (tap) {};
    \node[block, right of=f0, node distance=1.5cm] (f1) {$T_1$};
    \node[block, right of=z0, node distance=1.2cm] (z1) {$\zm$};
    \node[block, right of=f1, node distance=1.5cm] (f2) {$T_2$};
    \node[block, right of=z1, node distance=1.5cm] (z2) {$\zm$};
    \draw[->] (input) -- (f0);
    \draw[->] (tap.center) |- (z0);
    \draw[->] (z0) -| (f0);
    \draw[->] (f0) -- (f1);
    \draw[->] (z0) -- (z1);
    \draw[->] (z1) -| (f1);
    \draw[->] (f1) -- (f2);
    \draw[->] (z1) -- (z2);
    \draw[->] (z2) -| (f2);
    \node[right of=f2] (output) {$o$};
    \draw[->] (f2) -- (output);
\end{tikzpicture}
\end{center}

In practice, windowing is usually based on physical timestamps attached to
stream values rather than logical time.  For instance, the CQL~\cite{arasu-tr02} query
``\texttt{SELECT * FROM events [RANGE 1 hour]}'' returns all events received
within the last hour.  The corresponding circuit (on the left) takes input stream $s \in \stream{\Z[A]}$ and an additional
input $\theta \in \stream{\mathbb{R}}$ that carries the value of the current
time.

\begin{tabular}{m{3cm}m{0.5cm}m{3cm}}
\begin{tikzpicture}[>=latex]
    \node[] (input) {$s$};
    \node[above of=input, node distance=.5cm] (t) {$\theta$};
    \node[block, right of=input] (i) {$I$};
    \node[block, right of=i] (w) {$W$};
    \node[right of=w] (output) {$o$};
    \draw[->] (input) -- (i);
    \draw[->] (i) -- (w);
    \draw[->] (w) -- (output);
    \draw[->] (t) -| (w);
\end{tikzpicture}
&
$\cong$
&
\begin{tikzpicture}[>=latex]
    \node[] (input) {$s$};
    \node[above of=input, node distance=.5cm] (t) {$\theta$};
    \node[block, shape=circle, right of=input, inner sep=0pt] (plus) {$+$};
    \node[block, right of=plus] (w) {$W$};
    \node[right of=w] (output) {$o$};
    \node[block, below of=plus, node distance=.8cm] (z) {$\zm$};
    \draw[->] (input) -- (plus);
    \draw[->] (plus) -- (w);
    \draw[->] (t) -| (w);
    \draw[->] (w) -- node (mid) {} (output);
    \draw[->] (mid.center) |-  (z);
    \draw[->] (z) -- (plus);
\end{tikzpicture} \\
\end{tabular}

\noindent{}where the \emph{window operator} $W$ prunes input \zrs, only keeping values
with timestamps less than an hour behind $\theta[t]$.  Assuming $ts: A \to \mathbb{R}$ returns
the physical timestamp of a value, $W$ is defined as $W(v, \theta)[t] \defn \{x \in v[t] . 
ts(x) \geq \theta[t] - 1hr\}$.  Assuming $\theta$ increases monotonically, $W$
can be moved inside integration, resulting in the circuit on the right, which uses
bounded memory to compute a window of an unbounded stream.
This circuit is a building block of a large family of window queries, including
window joins and aggregation.  We conjecture that \dbsp can express 
any CQL query.

\begin{center}
\begin{tikzpicture}[>=latex]
  \node[] (input) {$i$};
  \node[right of=input] (invisible) {};
  \node[block, right of=invisible] (A) {$A$};
  \node[block, above of=invisible, node distance=.7cm] (S) {$S$};
  \node[block, circle, right of=S, inner sep=0cm] (m) {$-$};
  \node[block, circle, right of=A, inner sep=0cm] (p) {$+$};
  \node[right of=p, node distance=1.5cm] (output)  {$w$};
  \node[block, below of=p, node distance=.7cm] (z) {$\zm$};
  \draw[->] (input) -- (A);
  \draw[->] (input) |- (S);
  \draw[->] (A) -- (p);
  \draw[->] (p) -- node (mid) {} (output);
  \draw[->] (mid.center) |- (z);
  \draw[->] (S) -- (m);
  \draw[->] (m) -| (p);
  \draw[->] (z) -| (S);
  \draw[->] (z) -| (A);
\end{tikzpicture}
\end{center}

\subsection{Relational while queries}
(See also non-monotonic semantics for Datalog$^\neg$ and Datalog$^{\neg\neg}$\cite{Abiteboul-book95}.)
To illustrate the power of \dbsp we implement the following
``while'' program, where $Q$ is an arbitrary relational algebra query:
{\small
\begin{lstlisting}[language=Pascal]
x := i;
while (x changes)
    x := Q(x);
\end{lstlisting}}
The \dbsp implementation of this program is:

%$$\lambda i. \int[\D[\fix{\xi}{Q(\delta_0(i)+\zm(\xi))}]]$$
\begin{center}
\begin{tikzpicture}[>=latex]
  \node[] (input) {$i$};
  \node[block, right of=input] (delta) {$\delta_0$};
  \node[block, circle, right of=delta, inner sep=0cm] (p) {$+$};
  \node[block, right of=p] (Q) {$\lift Q$};
  \node[block, right of=Q] (D) {$\D$};
  \node[block, right of=D] (S) {$\int$};
  \node[right of=S] (output)  {$x$};
  \node[block, below of=p, node distance=.7cm] (z) {$\zm$};
  \draw[->] (input) -- (delta);
  \draw[->] (delta) -- (p);
  \draw[->] (p) -- (Q);
  \draw[->] (Q) -- node (mid) {} (D);
  \draw[->] (D) -- (S);
  \draw[->] (mid.center) |- (z);
  \draw[->] (S) -- (output);
  \draw[->] (z) -- (p);
\end{tikzpicture}
\end{center}

This circuit can be converted to a streaming circuit that computes a stream of values $i$ 
by lifting it; it can be incrementalized using Algorithm~\ref{algorithm-inc} to compute on changes of $i$:

\begin{center}
\begin{tikzpicture}[>=latex]
  \node[] (input) {$\Delta i$};
  \node[block, right of=input] (delta) {$\lift{\delta_0}$};
  \node[block, circle, right of=delta, inner sep=0cm] (p) {$+$};
  \node[block, right of=p] (Q) {$\inc{(\lift{\lift{Q}})}$};
  \node[block, right of=Q, node distance=1.5cm] (D) {$\lift{\D}$};
  \node[block, right of=D, node distance=1.1cm] (S) {$\lift{\int}$};
  \node[right of=S, node distance=1.2cm] (output)  {$\Delta x$};
  \node[block, below of=p, node distance=.9cm] (z) {$\lift{\zm}$};
  \draw[->] (input) -- (delta);
  \draw[->] (delta) -- (p);
  \draw[->] (p) -- (Q);
  \draw[->] (Q) -- node (mid) {} (D);
  \draw[->] (D) -- (S);
  \draw[->] (mid.center) |- (z);
  \draw[->] (S) -- (output);
  \draw[->] (z) -- (p);
\end{tikzpicture}
\end{center}

